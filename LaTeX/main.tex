
\documentclass[12pt,twoside,a4paper]{book}

\usepackage[T1]{fontenc}
\usepackage[utf8]{inputenc}
\usepackage[spanish]{babel}
\usepackage{graphicx}
\usepackage[dvips]{epsfig}
\usepackage{amssymb}

% Restore default bullet points
\renewcommand{\labelitemi}{\textbullet}
\renewcommand{\labelitemii}{\circ}
\renewcommand{\labelitemiii}{\diamond}
\renewcommand{\labelitemiv}{\cdot}

% Do not force pages to be filled vertically.
\raggedbottom

\usepackage{xurl}
\usepackage{hyperref}
\usepackage{lmodern}
\usepackage{amsmath}
\usepackage{eurosym}
\usepackage{float}
\usepackage{latexsym}
\usepackage{a4}
\usepackage{listings}
\usepackage{todonotes}
\usepackage{tabularx}
\usepackage{xltabular}
\newcommand{\code}[1]{\texttt{#1}}
\usepackage{hyperref}
\usepackage{makecell}
\usepackage{dirtree}
\usepackage[type={CC}, modifier={by-sa}, version={4.0}]{doclicense}
\usepackage{booktabs}
\usepackage{minted}

\usepackage[HTML]{xcolor}
\definecolor{codebg}{HTML}{FCFCFC}

% Diagrams directly in the document.
\usepackage{tikz}
\usetikzlibrary{shapes.arrows, shapes.geometric, shapes.symbols, shapes.misc, arrows.meta, positioning, fit, backgrounds, calc, shadows, babel}

\usepackage[minted]{tcolorbox}
\newtcblisting{huggedminted}[2][]{%
    listing engine=minted,
    minted language=#2,
    minted options={fontsize=\footnotesize, #1},
    hbox,           % Hug contents.
    colback=codebg,  % Background color
    colframe=codebg, % Frame color
    boxrule=0pt,    % No frame
    left=2pt, right=2pt, top=2pt, bottom=2pt,
    arc=0pt,
    listing only    % Contains only the code
}

% Display "+" and "-" like regular characters.
\setminted{ignorelexererrors=true}

% Custom indexes.
\usepackage{tocloft}

\newcommand{\reqlist}{Índice de requisitos}
\newlistof{requirements}{req}{\reqlist}

% Adjust width for the "RF-XX" labels
\setlength{\cftrequirementsnumwidth}{5em} 

\newcounter{frcounter}
\renewcommand{\thefrcounter}{%
  \ifnum\value{frcounter}<10 0\fi\arabic{frcounter}%
}
\newcommand{\freq}[6]{%
  \refstepcounter{frcounter}%
  \addcontentsline{req}{requirements}{\protect\numberline{\textbf{RF-\thefrcounter}}#1}%
  
  \subsubsection*{RF-\thefrcounter: #1}
  
  % Create automatic label: "req:RF-XX"
  \edef\currentreqlabel{req:RF-\thefrcounter}
  \label{\currentreqlabel}

  \textbf{Descripción:} #2

  \par\medskip
  \textbf{Prioridad:} #3

  \par\medskip
  \textbf{Entradas:} #4

  \par\medskip
  \textbf{Salidas:} #5

  \par\medskip
  \textbf{Notas:} #6

  \par\bigskip
}

\newcounter{nfrcounter}
\renewcommand{\thenfrcounter}{%
  \ifnum\value{nfrcounter}<10 0\fi\arabic{nfrcounter}%
}
\newcommand{\nfreq}[4]{%
  \refstepcounter{nfrcounter}%
  \addcontentsline{req}{requirements}{\protect\numberline{\textbf{RNF-\thenfrcounter}}#1}%
  
  \subsubsection*{RNF-\thenfrcounter: #1}
  
  % Create automatic label: "req:RNF-XX"
  \edef\currentreqlabel{req:RNF-\thenfrcounter}
  \label{\currentreqlabel}

  \textbf{Descripción:} #2

  \par\medskip
  \textbf{Prioridad:} #3

  \par\medskip
  \textbf{Criterio de verificación:} #4

  \par\bigskip
}

\newcounter{ircounter}
\renewcommand{\theircounter}{%
  \ifnum\value{ircounter}<10 0\fi\arabic{ircounter}%
}
\newcommand{\ireq}[2]{%
  \refstepcounter{ircounter}%
  \addcontentsline{req}{requirements}{\protect\numberline{\textbf{RI-\theircounter}}#1}%
  
  \subsubsection*{RI-\theircounter: #1}
  
  % Create automatic label: "req:RI-XX"
  \edef\currentreqlabel{req:RI-\theircounter}
  \label{\currentreqlabel}

  \textbf{Descripción:} #2

  \par\bigskip
}

\begin{document}
\pagestyle{empty}
\begin{center}
	{\bf\Large UNIVERSIDAD DE SANTIAGO DE COMPOSTELA}
	
	\vspace{1cm}
	\includegraphics[width=5cm]{figuras/logo_usc.eps}
	
	\vspace{1cm}
	{\bf\large ESCUELA TÉCNICA SUPERIOR DE INGENIERÍA}
	
	\vspace{2cm}
	{\bf\LARGE Consulta eficiente de redes geoespaciales}
	
	\vspace{0.5cm}
	{\bf\large Una comparativa entre BBDD de grafos y objeto-relacionales}
\end{center}

\vspace{2cm}
\hspace{4cm}\begin{tabular}{l}
	{\it\Large Autor:} \\
	{\bf\Large Daniel Ramos Rodríguez} \\
	~ \\
	{\it\Large Tutor:} \\
	{\bf\Large José Ramón Ríos Viqueira} \\
\end{tabular}

\vspace{2cm}
\begin{center}
	{\bf\Large Grado en Ingeniería Informática}
	
	\vspace{0.5cm}
	{\bf\large Febrero 2026}
	
	\vspace{0.5cm}
    Trabajo de Fin de Grado presentado en la Escuela Técnica Superior de Ingeniería de la Universidad de Santiago de Compostela para la obtención del Grado en Ingeniería Informática
\end{center}

\cleardoublepage
\include{capitulos/certificacion} % paxina de certificación (optativa)
\cleardoublepage

\setlength{\parskip}{6pt}
\pagestyle{plain}
\chapter*{Resumen}
En este trabajo se han modelado redes geoespaciales de transporte público en dos sistemas gestores de bases de datos diferentes, PostgreSQL y Neo4J, partiendo del estándar de modelado GTFS. Después, se ha desarrollado un conjunto representativo de consultas que responden a necesidades de información reales, tanto de usuarios de la red como de las empresas que ofrecen dichos servicios.

A continuación, se ha trabajado en la representación gráfica, sobre el globo terráqueo, de los resultados generados, utilizando QGIS y Folium, y, en último lugar, se ha llevado a cabo una comparativa entre las prestaciones de ambos sistemas, centrada no solo en la eficiencia computacional, sino también en la facilidad de modelado y expresión de consultas.
 

\cleardoublepage
\pagestyle{plain}
\addto\captionsspanish{\def\contentsname{Memoria tipo B -- \'{I}ndice general }}
\tableofcontents
\listoffigures
\listoftables
\listofrequirements

% Agora incluimos os capítulos. Cambiamos a numeración e as cabeceiras
\cleardoublepage
\pagenumbering{arabic}
\setcounter{page}{1}
\pagestyle{headings}

\chapter{Introducción}

La gestión y el análisis de datos almacenados en redes geoespaciales se han convertido en parte crucial de múltiples dominios, destacando especialmente en el sector del transporte público. A causa de la creciente complejidad de dichas redes, que combinan información geográfica y temporal, se vuelve necesario el desarrollo de soluciones robustas y eficientes, que a su vez sean capaces de representar de manera conveniente las relaciones entre elementos como paradas, rutas y horarios.

En este contexto, las tecnologías de bases de datos desempeñan un papel crucial. Aunque los sistemas gestores de bases de datos relacionales, es decir, SQL, han sido tradicionalmente la opción dominante por un amplio margen, el surgimiento de los nuevos paradigmas NoSQL, en particular las bases de datos orientadas a grafos, ofrece enfoques alternativos. Se parte de la hipótesis de que estas últimas podrían ofrecer ventajas significativas en eficiencia y expresividad de consultas, al estar centradas en el modelado de un gran número de relaciones y la recuperación de información mediante la navegación a través de nodos.

Este trabajo aborda precisamente el reto de gestionar redes geoespaciales de transporte público, explorando y comparando dos aproximaciones tecnológicas diferentes: un sistema relacional consolidado como PostgreSQL, junto a su extensión geoespacial PostGIS, frente a una de las opciones mejor establecidas en el ámbito de las bases de datos de grafos, Neo4J, utilizando la extensión Neo4J Spatial. En ambos casos se ha optado por tecnologías estables y maduras para garantizar la fiabilidad de los sistemas.

En cuanto a los conjuntos de datos empleados, estos siguen el estándar de facto \textbf{GTFS} (General Transit Feed Specification), y representan redes a diferentes escalas, incluyendo tanto entornos urbanos (Bogotá, Praga, Hong Kong...), como regionales (por ejemplo, Galicia), a fin de poder considerar la escalabilidad que cada sistema ofrece.

Dado que no existe una manera simple de trasladar la información, se busca desarrollar mecanismos para \textbf{importar y modelar} las redes en ambos sistemas gestores de bases de datos y, posteriormente, implementar un conjunto representativo de \textbf{consultas} pensadas para satisfacer necesidades de información típicas, ya sean del usuario final o de las entidades gestoras del transporte público.

Una vez recuperada la información necesaria, el siguiente paso es presentar los resultados de manera amigable a los usuarios, y, para lograr esto, proporcionar una solución para la \textbf{visualización} de los mismos, a través del software QGIS y de la biblioteca Folium, de Python, facilitando su interpretación sobre representaciones cartográficas.

En último lugar, se realiza una \textbf{evaluación comparativa} de los dos sistemas, que no se limita a métricas cuantitativas, es decir, el rendimiento, sino que también incluye apartados cualitativos como la facilidad para modelar la red geoespacial y la comodidad a la hora de expresar las consultas en el lenguaje de cada sistema. Con ello, se busca determinar las fortalezas e inconvenientes de cada una de las alternativas y clarificar en qué ocasiones resulta preferible su uso.

A lo largo del resto de la memoria se profundiza en diversos aspectos del trabajo. Tras esta introducción, se definen las necesidades de información y funcionalidades relevantes para los usuarios y gestores del sistema de transporte público (capítulo \ref{chapter:requirements}), se detalla la solución propuesta, tanto a nivel de modelado como de importación y consulta (capítulo \ref{chapter:design}), se documenta el plan de pruebas seguido para validar los requisitos previamente planteados (capítulo \ref{chapter:tests}), se realiza una evaluación comparativa tanto del rendimiento como de la facilidad de expresión de consultas en cada uno de los sistemas (capítulo \ref{chapter:comparison}) y, finalmente, se presentan las conclusiones obtenidas a raíz de la comparación entre ambos sistemas, así como líneas de trabajo futuras para ampliar el alcance del proyecto (capítulo \ref{chapter:conclusion}).

Por otra parte, la memoria se complementa con tres anexos: el manual técnico, para las personas que modifiquen el sistema o lleven a cabo algún desarrollo posterior (anexo \ref{appendix:development}), el manual de usuario, que explica cómo instalar, configurar y utilizar el software (anexo \ref{appendix:usage}), y la licencia (anexo \ref{appendix:license}).

\cleardoublepage

\chapter{Especificación de requisitos} \label{chapter:requirements}

En este capítulo se proporciona una descripción detallada de los requisitos del sistema, tanto a nivel funcional como no funcional y de información.

\section{Alcance}

El sistema \textbf{permitirá}:
\begin{itemize}
    \item Cargar conjunto de datos que siga el formato GTFS, incluyendo sus apartados opcionales, pero solamente en formato CSV.
    \item Ejecutar un conjunto predefinido de consultas relevantes sobre los datos cargados, así como consultas escritas por los propios usuarios.
    \item Visualizar de forma básica la red así como los resultados de ciertas consultas espaciales.
    \item Comprobar el tiempo de importación de los conjuntos de datos y de ejecución de las consultas.
\end{itemize}

El sistema \textbf{no permitirá}:
\begin{itemize}
    \item Cargar un conjunto de datos en formato GeoJSON.
    \item Modificar los datos de manera interactiva una vez cargados.
    \item Distinguir múltiples usuarios o roles.
    \item Obtener el máximo rendimiento que pueda ofrecer el sistema gestor de bases de datos.
\end{itemize}


\section{Requisitos funcionales}

% Import and validation.
\freq
    {Importación de datos}
    {El sistema debe permitir importar un conjunto de datos que siga el estándar GTFS\cite{gtfs}, de acuerdo con su versión de enero de 2025, con todos sus archivos en formato CSV y sin campos faltantes.}
    {Máxima.}
    {Nombre del conjunto de datos a importar, ya ubicado en el directorio de datasets del sistema.}
    {Mensaje de confirmación o error, informando al usuario del resultado.}
    {Aunque no pueden faltar campos en los archivos, se admiten valores nulos.}

\freq
    {Validación de integridad y restricciones}
    {El sistema debe asegurar que los datos importados cumplen con las restricciones definidas por el estándar GTFS, lo que incluye verificación de tipos, presencia de campos obligatorios, e integridad referencial entre las entidades.}
    {Alta.}
    {Conjunto de datos que está siendo importado.}
    {En caso de error, mensaje que señale las restricciones no satisfechas por el conjunto de datos.}
    {Este proceso no requiere interacción por parte del usuario, sino que se lleva a cabo de forma automática durante la importación.}
    
% Queries.
\freq
    {Servicios de transporte activos}
    {El sistema debe ser capaz de determinar los servicios de transporte activos para una fecha dada.}
    {Alta.}
    {Fecha en la que realizar la búsqueda.}
    {Lista de identificadores de servicios activos.}
    {—}

\freq
    {Estadísticas diarias}
    {El sistema debe ser capaz de calcular el número de trayectos, rutas y paradas activas en la red de transporte para un rango de fechas dado.}
    {Media.}
    {Fechas inicial y final del rango.}
    {Lista de trayectos, rutas y paradas activas para cada uno de los días del rango.}
    {—}

\freq
    {Horas de salida de trayectos}
    {El sistema debe ser capaz de encontrar las horas de salida para todos los trayectos vinculados a una ruta, desde una parada concreta y en una fecha especificada.}
    {Alta.}
    {Identificadores de ruta y parada, fecha de consulta.}
    {Lista de trayectos, junto a su destino, y horas de salida, ordenada temporalmente.}
    {—}

\freq
    {Tiempos de espera en la red}
    {El sistema debe ser capaz de calcular, para todas las rutas, los tiempos de espera mínimos, promedio y máximos, así como la desviación estándar, agregando estos a lo largo de cada una de las paradas, para una fecha concreta.}
    {Media.}
    {Fecha de consulta.}
    {Lista de tiempos de espera mínimos, promedio y máximos, junto a su desviación estándar, para cada una de las rutas.}
    {—}

\freq
    {Próximas salidas}
    {El sistema debe ser capaz de determinar, para una parada concreta, y una fecha y hora dadas, los próximos trayectos que parten desde la misma, ordenados por tiempo de salida.}
    {Alta.}
    {Identificador de parada, fecha y hora de consulta.}
    {Lista de próximos trayectos, con ruta, destino y hora de salida, ordenados por esta última.}
    {—}

\freq
    {Segmentos con solapamiento}
    {El sistema debe ser capaz de analizar qué pares de paradas consecutivos son cubiertos por un mayor número de rutas.}
    {Media.}
    {—}
    {Lista de pares de paradas, número y nombre de rutas que los cubren, ordenada por el número de rutas de manera descendente.}
    {El interés de esta consulta reside en que permite determinar planificación redundante en la red de transporte.}

\freq
    {Ranking de rutas por relevancia}
    {El sistema debe ser capaz de calcular, dadas una fecha y hora, las rutas más importantes en base al número de trayectos activos y la frecuencia de estos.}
    {Baja.}
    {Fecha y hora de consulta.}
    {Lista de rutas ordenada de manera descendente por número de trayectos y de manera ascendente por la frecuencia de los mismos.}
    {—}

\freq
    {Ranking de rutas por velocidad}
    {El sistema debe ser capaz de encontrar las rutas más rápidas de la red, considerando la agregación de la velocidad promedio de cada uno de sus trayectos.}
    {Media.}
    {—}
    {Lista de rutas con velocidad promedio y número de trayectos agregados para el cálculo, ordenada de manera descendente por dicha velocidad.}
    {—}

\freq
    {Paradas cercanas}
    {El sistema debe ser capaz de hallar las paradas más cercanas a una ubicación proporcionada por el usuario, dada una distancia de búsqueda.}
    {Alta.}
    {Latitud y longitud de la localización y radio de búsqueda.}
    {Lista de paradas próximas, junto a sus latitudes, longitudes y distancias respecto al punto dado, ordenada por estas últimas.}
    {—}

\freq
    {Paradas más transitadas}
    {El sistema debe ser capaz de determinar las paradas más importantes en la red de transportes para una fecha dada, de acuerdo con el número de trayectos que parten de ellas.}
    {Baja.}
    {Fecha de consulta.}
    {Lista de paradas con el número y nombre de rutas que las atraviesan, el número de trayectos que parten de ellas y la hora de la primera y última salidas, ordenada por el número de trayectos de manera descendente.}
    {—}

\freq
    {Histograma de comienzo de trayectos}
    {El sistema debe ser capaz de generar, dada una longitud de intervalo y una fecha, un histograma que muestre cuántos trayectos comienzan en cada intervalo temporal, a lo largo del día.}
    {Alta.}
    {Longitud de intervalo en minutos y fecha de consulta.}
    {Lista de intervalos en que se divide la jornada y número de trayectos que inician en cada uno de ellos, ordenada de manera temporal.}
    {—}

\freq
    {Mapa de calor de paradas}
    {El sistema debe ser capaz de dividir de forma geométrica la región sobre la que se encuentra la red de transportes y, para cada una de esas subdivisiones, determinar el número de paradas contenidas.}
    {Baja.}
    {Tamaño de las subdivisiones.}
    {Lista de geometrías (cada una de ellas representando una subdivisión) y número de paradas que se encuentran en cada una.}
    {La implementación de este requisito está supeditada al soporte proporcionado por las bibliotecas PostGIS y Neo4J Spatial. En ningún caso se implementará el algoritmo de segmentación de manera manual.}

\freq
    {Análisis de alcanzabilidad}
    {El sistema debe ser capaz de encontrar, a partir de una parada inicial, la hora mínima de llegada para cada una de las otras paradas del sistema, considerando no solo los trayectos de la red de transporte, sino también los transbordos entre estaciones y la posibilidad de caminar entre paradas.}
    {Alta.}
    {Identificador de la parada inicial, fecha y hora de partida.}
    {Lista de paradas de la red, hora mínima de llegada a cada una de ellas, parada previa en el viaje, y acción realizada desde la parada previa (trayecto, transbordo o caminar).}
    {Dado que el resultado varía en función tanto de elementos dependientes del tiempo (trayectos) como independientes (transbordos, desplazamiento a pie), es probable que el soporte de bibliotecas no resulte suficiente y sea necesario implementar un algoritmo propio.}

\freq
    {Camino más corto}
    {El sistema debe ser capaz de encontrar, a partir de una parada de partida, el camino que le permite llegar a otra de destino de la manera más rápida posible, considerando una fecha y hora de inicio concretas.}
    {Alta.}
    {Identificador de las paradas de partida y destino, fecha y hora de inicio.}
    {Lista de paradas a recorrer, así como la hora de llegada a cada una de ellas y la acción tomada en la parada previa (trayecto, transbordo o caminar).}
    {—}

% Visualization.
\freq
    {Visualización espacial de resultados}
    {El sistema debe permitir al usuario generar mapas que representen los resultados espaciales producidos por la base de datos (por ejemplo, un mapa de calor sobre la densidad de paradas en que, al colocar el ratón sobre una subdivisión del mapa, se muestre el número concreto).}
    {Media.}
    {Datos requeridos por la consulta cuyo resultado se quiera visualizar.}
    {Mapa interactivo que muestre el resultado de la consulta.}
    {—}
    
\freq
    {Generación de gráficas}
    {El sistema debe permitir al usuario, en aquellos casos en que la consulta sea de índole temporal, generar una representación visual no interactiva (por ejemplo, una gráfica de mancuerna para los tiempos de espera en cada ruta).}
    {Alta.}
    {Datos requeridos por la consulta cuyo resultado se quiera visualizar.}
    {Gráfica que muestre el resultado de la consulta.}
    {—}

% Measurements.
\freq
    {Medición de rendimiento}
    {El sistema debe permitir al usuario medir el tiempo de ejecución tanto del proceso de importación como de las consultas predefinidas.}
    {Media.}
    {Conjunto(s) de datos a importar para evaluar el rendimiento de la importación, o consulta(s) a ejecutar en caso contrario.}
    {Lista de tiempos de ejecución para la acción realizada.}
    {—}


\section{Requisitos no funcionales}

\nfreq
    {Escalabilidad respecto al conjunto de datos}
    {El sistema debe poder importar conjuntos de datos GTFS de hasta 500 MB de tamaño en texto plano, utilizando hardware no especializado (ordenadores personales, con 16 GB de RAM) sin incurrir en errores de desbordamiento de memoria (OOM).}
    {Alta.}
    {Importación de un conjunto de datos GTFS de aproximadamente 500 MB.}

\nfreq
    {Extensibilidad del catálogo de consultas}
    {El sistema debe permitir añadir nuevas consultas predefinidas introduciendo únicamente el código SQL o Cypher de dichas consultas en los ficheros correspondientes, sin necesidad de refactorizar el código.}
    {Alta.}
    {Adición de una nueva consulta introduciendo únicamente su declaración (nombre y argumentos) y definición (algoritmo), sin modificar o borrar código previamente existente.}

\nfreq
    {Portabilidad de las visualizaciones}
    {El sistema debe garantizar que las visualizaciones generadas utilizan formatos portables (PNG en el caso de las gráficas, HTML en el de los mapas interactivos).}
    {Media.}
    {Apertura de archivos generados en aplicaciones de escritorio estándar (visor de imágenes y navegador).}

\nfreq
    {Interoperabilidad con estándares geoespaciales (GIS)}
    {El sistema debe almacenar la información geoespacial de acuerdo con los estándares definidos por el Open Geospatial Consortium, como, por ejemplo, utilizando los tipos WKB y WKT.}
    {Alta.}
    {Conexión desde una herramienta GIS externa como QGIS, realización de una consulta de prueba y procesamiento por parte de la herramienta de los datos devueltos por la BBDD.}

\nfreq
    {Interfaces de línea de comandos (CLI)}
    {El sistema debe exponer su funcionalidad a través de herramientas (scripts) ejecutados en la línea de comandos, tanto para el proceso de importación como para la generación de visualizaciones, tomando como argumentos las entradas necesarias (nombre del dataset o argumentos de la consulta).}
    {Media.}
    {Ejecución en un terminal de los scripts de importación y producción de visualizaciones.}

\section{Requisitos de información}

\ireq
    {Entidades del estándar GTFS}
    {El sistema debe almacenar y gestionar toda la información definida en la especificación GTFS (Static), incluyendo, entre otros, agencias y servicios de transporte, paradas, rutas, trayectos, tarifas y datos de accesibilidad, preservando el componente semántico de dicha información, pero no necesariamente la representación.}

\ireq
    {Estructuras espaciales derivadas}
    {El sistema debe generar y mantener representaciones enriquecidas de los datos originales para su análisis, y, en particular, objetos geométricos sobre la superficie terrestre (puntos asociados a la ubicación de las paradas, líneas que describan el trazado de cada trayecto, y polígonos que delimiten las zonas de tarificación consideradas por los servicios de transporte).}

\ireq
    {Métricas de rendimiento}
    {El sistema debe almacenar los resultados cuantitativos de las pruebas de rendimiento, siendo estos fundamentalmente los tiempos de importación de los conjuntos de datos y de ejecución de las consultas.}

\cleardoublepage

\chapter{Diseño} \label{chapter:design}

Una vez definido el sistema a construir, este capítulo aborda el diseño del mismo, haciendo uso de una aproximación \textit{top-down}, es decir, proporcionando en primer lugar una visión global de la arquitectura y profundizando progresivamente en los detalles de implementación.

\section{Arquitectura del sistema}

En términos generales, el sistema está diseñado siguiendo una \textbf{arquitectura modular} y desacoplada, donde un controlador orquesta el flujo de datos y el proceso de pruebas. Adicionalmente, y como se muestra en la figura \ref{design:architecture}, el sistema está compuesto por tres capas:

\begin{enumerate}
    \item \textbf{Control:} Incluye el ya mencionado controlador, que procesa los conjuntos de datos antes de su importación, coordina las transacciones en las bases de datos y gestiona la ejecución de las pruebas. Para la comunicación, utiliza \texttt{psycopg} y \texttt{neo4j-driver} sobre TCP/IP.
    \item \textbf{Persistencia:} Compuesta por los motores de base de datos, PostgreSQL\cite{postgres} y Neo4J\cite{neo4j}, que se despliegan como contenedores Docker aislados, y operan de manera independiente entre sí.
    \item \textbf{Presentación:} Abarca tanto la integración con herramientas externas para la visualización (QGIS), como los artefactos generados para su análisis, en forma de mapas interactivos o gráficas.
\end{enumerate}

\begin{figure}[ht]
    \centering
    \begin{tikzpicture}[
        % Define styles
        font=\sffamily\footnotesize,
        node distance=1.5cm and 2cm,
        % Main component style
        component/.style={
            rectangle, 
            draw=black!60, 
            fill=white, 
            very thick, 
            rounded corners=3pt, 
            minimum width=2.5cm, 
            minimum height=1cm, 
            align=center,
            drop shadow
        },
        % Database style
        database/.style={
            cylinder, 
            cylinder uses custom fill, 
            cylinder body fill=gray!10, 
            cylinder end fill=gray!30, 
            shape border rotate=90, 
            aspect=0.25, 
            draw=black!60, 
            thick, 
            minimum width=1.5cm, 
            minimum height=1.5cm,
            align=center
        },
        % File/Artifact style
        artifact/.style={
            tape, 
            tape bend top=none, 
            draw=black!60, 
            thick, 
            fill=yellow!10, 
            minimum width=2cm, 
            minimum height=1.2cm, 
            align=center
        },
        % Connection style
        link/.style={
            -{Latex[length=3mm]}, 
            thick, 
            draw=black!70
        },
        % Label style for arrows
        arrowlabel/.style={
            midway,
            above,
            text=black!70, 
            font=\sffamily\tiny, 
            inner sep=1.5pt
        }
    ]

    % Nodes
    % Central Controller (Python)
    \node[component, fill=green!10] (python) {
        \textbf{Controlador} \\
        \textit{(Python 3.13)} \\
        \texttt{pytest}, \texttt{pandas}, \texttt{folium}
    };
    % Inputs (Left)
    \node[artifact, left=of python] (inputs) {
        \textbf{Datasets GTFS} \\
        \textit{(.txt / .csv)}
    };
    % Outputs (Right)
    \node[artifact, right=of python, fill=orange!10] (outputs) {
        \textbf{Resultados} \\
        \textit{Mapas (HTML)} \\
        \textit{Gráficas (PNG)} \\
        \textit{Métricas (JSON)}
    };

    % Infrastructure Group (Bottom)
    % We use a coordinate to help positioning
    \coordinate[below=2.8cm of python] (infra_center);
    % Neo4j Container
    \node[database, left=0.5cm of infra_center, text width=1.5cm] (neo4j) {Neo4J};
    % PostgreSQL Container
    \node[database, right=0.5cm of infra_center, text width=1.5cm] (postgres) {Postgres};
    % Docker Boundary Box
    \begin{scope}[on background layer]
        \node[
            fit=(postgres) (neo4j), 
            draw=blue!30, 
            dashed, 
            thick, 
            fill=blue!5, 
            rounded corners, 
            inner sep=0.5cm,
            label={[blue!50, anchor=north]south:\textbf{Docker (localhost)}}
        ] (docker_env) {};
    \end{scope}

    % QGIS
    \node[component, right=2cm of postgres, fill=cyan!10] (qgis) {
        \textbf{Visor GIS} \\
        \textit{(QGIS)}
    };

    % Connections
    % Python -> Input
    \draw[link] (python) -- node[arrowlabel] {Preprocesado} (inputs);

    % Python -> Output
    \draw[link] (python) -- node[arrowlabel] {Generación} (outputs);

    % Python <-> Neo4J
    \draw[link, <->] ([xshift=-0.5cm]python.south) -- node[arrowlabel, sloped] {\texttt{neo4j-driver}} (neo4j.north);

    % Python <-> PostgreSQL
    \draw[link, <->] ([xshift=0.5cm]python.south) -- node[arrowlabel, sloped] {\texttt{psycopg}} (postgres.north);

    % Input -> Docker
    \draw[link] (inputs.south) |- node[arrowlabel, near end] {Importación} (docker_env.west);

    % Docker -> QGIS
    \draw[link] (postgres.east) -- node[arrowlabel] {Visualización} (qgis.west);

    \end{tikzpicture}
    \caption{Diagrama de la arquitectura general del sistema.}
    \label{design:architecture}
\end{figure}

\section{Infraestructura y despliegue}

\subsection{Aislamiento y reproducibilidad}

Para garantizar que los componentes de la capa de persistencia, es decir, los motores de base de datos, no interactúen entre sí, generando incompatibilidades, estos se virtualizan mediante contenedores Docker\cite{docker}. Además, de esta forma se logra la reproducibilidad e independencia del sistema operativo y bibliotecas subyacentes, al encapsular las dependencias de cada motor y sus extensiones geoespaciales.

Las imágenes Docker seleccionadas son \texttt{pgrouting:17-3.5-main}, en el caso de PostgreSQL, que incluye tanto la extensión \texttt{postgis}, como \texttt{pgrouting}, y en el de Neo4J, \texttt{neo4j:5.19.0-enterprise}, en lugar de la versión Community, para poder hacer uso de, entre otros, restricciones de tipos y existencia de propiedades.

No obstante, a diferencia de las arquitecturas de microservicios convencionales, no se considera necesario hacer uso de Docker Compose, optando por scripts de lanzamiento de cada base de datos de manera independiente, dado que los motores no necesitan comunicarse entre sí (\textit{shared-nothing architecture}), e introducir un fichero \texttt{docker-compose.yml} no aportaría ningún beneficio, y forzaría al usuario a lanzar ambos motores incluso aunque solo vaya a utilizar uno, desperdiciando recursos del sistema.

\subsection{Conexiones}

Una de las responsabilidades principales del controlador Python\cite{python} es gestionar el ciclo de vida de las conexiones, durante tanto el proceso de prueba como la explotación del sistema, mediante el uso de \textit{fixtures} proporcionadas por \texttt{pytest}.

Para ello, se sigue el flujo descrito en la figura \ref{design:connections}, lo que permite garantizar al mismo tiempo que el sistema no desperdicia recursos abriendo y cerrando más conexiones de las necesarias y que, cuando termina la ejecución de consultas o \textit{statements}, los sockets son liberados correctamente, sin generar \textit{leaks}.

\begin{figure}[ht]
    \centering
    \begin{tikzpicture}[
        font=\sffamily\footnotesize,
        node distance=1.0cm,
        process/.style={
            rectangle, 
            draw=black!70, 
            thick, 
            fill=white, 
            rounded corners=3pt, 
            minimum width=4.5cm, 
            minimum height=1cm, 
            align=center,
            drop shadow
        },
        decision/.style={
            diamond, 
            draw=black!70, 
            thick, 
            fill=yellow!10, 
            aspect=2, 
            align=center,
            inner sep=0pt,
            minimum width=2.5cm,
            drop shadow
        },
        state/.style={
            rectangle, 
            rounded corners=8pt,
            draw=black!60, 
            thick, 
            fill=gray!10, 
            minimum width=2.5cm, 
            align=center,
            font=\sffamily\scriptsize\itshape
        },
        arrow/.style={
            -{Latex[length=3mm]}, 
            thick, 
            draw=black!70,
            rounded corners=5pt
        },
        label_text/.style={
            font=\sffamily\tiny\bfseries, 
            text=black!70,
            inner sep=2pt,
            midway
        }
    ]

    % Precondition
    \node[state] (precondition) {Infraestructura Docker (en ejecución)};

    % Step 1: Setup
    \node[process, below=1cm of precondition, fill=green!10] (connect) {
        \textbf{1. Establecimiento de conexión TCP/IP} \\
        \textit{Drivers:} \texttt{psycopg}, \texttt{neo4j-driver}
    };

    % Step 2a: Execution
    \node[process, below=1cm of connect, fill=blue!10] (execute) {
        \textbf{2. Ejecución mediante \textit{Query Runners}} \\
        Envío de consulta / statement
    };

    % Step 2b: Decision
    \node[decision, below=1cm of execute] (decide) {
        ¿Más\\operaciones?
    };

    % Step 3: Teardown
    \node[process, below=1cm of decide, fill=red!10] (close) {
        \textbf{3. Cierre de conexión} \\
        Liberación de sockets
    };

    % Fixture scope
    \begin{scope}[on background layer]
        \node[
            fit=(connect) (execute) (decide) (close), 
            draw=black!40, 
            dashed, 
            thick, 
            fill=gray!5, 
            rounded corners=5pt,
            inner xsep=2cm,
            inner ysep=0.5cm,
            label={[anchor=north]south:\textbf{Fixture}}
        ] (scope) {};
    \end{scope}

    % Arrows
    \draw[arrow] (precondition) -- (scope.north);
    \draw[arrow] (connect) -- (execute);
    \draw[arrow] (execute) -- (decide);
    \draw[arrow] (decide.south) -- node[label_text, right=4pt] {No} (close.north);
    \draw[arrow] (decide.east) -- ++(2.5,0) |- node[label_text, right=4pt, near start] {Sí} (execute.east);
    
    \end{tikzpicture}
    \caption{Ciclo de vida de las conexiones del sistema.}
    \label{design:connections}
\end{figure}

Cabe mencionar que el flujo presentado se aplica únicamente al controlador Python y, si se opta por utilizar un software GIS de escritorio, como QGIS, este se encarga de establecer y administrar la conexión con PostgreSQL, de forma independiente, y a partir ciertos parámetros proporcionados, como el \textit{endpoint} y las credenciales de acceso.

\subsection{Requisitos}

\subsubsection{Hardware}

Para asegurar que las métricas de rendimiento del sistema reflejan las capacidades de los motores de bases de datos y no las limitaciones del hardware subyacente, se han establecido los siguientes requisitos mínimos:

\begin{itemize}
    \item \textbf{Procesador:} Intel i7 de octava generación, o su equivalente en AMD (arquitectura x86-64), con soporte para virtualización, necesaria para la ejecución de los contenedores Docker.
    \item \textbf{RAM:} 16 GB, dado que tanto la importación como las consultas cargan una gran cantidad de datos en memoria, especialmente en Neo4J, al almacenar la topología de la red, y se busca evitar el \textit{swapping}.
    \item \textbf{Disco duro:} Un disco de estado sólido (SSD) con al menos 10 GB libres, dado que el proceso de importación es intensivo en E/S, por lo que un disco duro mecánico (HDD) se convertiría en un cuello de botella.
\end{itemize}

Por otra parte, para la obtención de los resultados presentados en el capítulo \ref{chapter:comparison}, los requisitos previamente listados se materializan en un sistema con un procesador Intel i7 10750H, 32 GB de RAM y un SSD con interfaz NVMe.

\subsubsection{Software}

En cuanto al entorno software, las dependencias principales son:

\begin{itemize}
    \item \textbf{Sistema operativo:} Cualquier distribución de Linux, para poder ejecutar Docker de manera nativa.
    \item \textbf{Python y bibliotecas:} Versión 3.13 del lenguaje, junto con las bibliotecas listadas en el fichero \texttt{requirements.txt} del directorio del proyecto: entre otras, \texttt{pytest}\cite{pytest} para la automatización de las pruebas, \texttt{folium}\cite{folium} para la generación de mapas interactivos y \texttt{pandas}\cite{pandas} para el preprocesado de los conjuntos de datos, así como los drivers de PostgreSQL y Neo4J.
    \item \textbf{Visor GIS:} Cualquier visor GIS que soporte conexiones con PostgreSQL. Se recomienda QGIS\cite{qgis} dado que es el estándar de la industria.
\end{itemize}

\section{Principios y patrones de diseño}

La construcción del sistema no se limita exclusivamente a la codificación, sino que también incorpora principios arquitectónicos y patrones de diseño, a fin de garantizar la modularidad, la integridad de los datos y también el mantenimiento de los componentes software.

En este apartado, se presentan algunos de los principios y patrones aplicados, partiendo de la arquitectura y concretando progresivamente en aspectos del código.

\subsection{Separación de responsabilidades}

La definición del modelo de datos se encuentra desacoplada de la implementación de consultas, y estas, a su vez, de la orquestación del sistema. Gracias a ello, la modificación de alguno de los componentes no afecta al resto.

\subsection{KISS \textit{(Keep it simple, stupid)}}

Se opta por utilizar las bibliotecas \texttt{psycopg} y \texttt{neo4j-driver} de manera directa, en lugar de escoger un ORM pesado como SQLAlchemy, lo que reduce el consumo de recursos sin un impacto tangible en la funcionalidad.

Por otra parte, dado que uno de los objetivos de este trabajo es llevar a cabo una comparativa de rendimiento, esto también garantiza la \textbf{transparencia} de las mediciones, asegurando que se observa el rendimiento ofrecido por los motores de base de datos y no el que proporciona el framework escogido.

\subsection{DRY \textit{(Don't repeat yourself)}}

En la implementación de las pruebas, el fichero \texttt{conftest.py} se encarga de gestionar los detalles comunes, como el flujo de ejecución de los casos de prueba, análogo para todos ellos, evitando la repetición de código y favoreciendo la mantenibilidad.

\subsection{Defensa en profundidad}

Para evitar que el sistema opere sobre datos semánticamente incorrectos, no solamente se preprocesan los \textit{datasets} y se definen las queries considerando valores atípicos, sino que también se lleva a cabo la \textbf{aplicación estricta del esquema}.

Mediante restricciones nativas a los motores de base de datos, (\texttt{CHECK}, \texttt{NOT NULL}, \textit{triggers}, limitaciones de tipos), se verifica que los datos estén presentes si son imprescindibles, que se encuentren en los rangos de valores aceptados por el estándar GTFS, y que se almacenen en un formato estandarizado (por ejemplo, todos los colores son cadenas de 6 dígitos hexadecimales, en mayúsculas).

De esta forma, se genera un \textbf{\textit{pipeline} de defensa en profundidad}, o \textit{pipeline} de validación, tal y como se muestra en la figura \ref{design:layered_defense}.

\begin{figure}[ht]
    \centering
    \begin{tikzpicture}[
        node distance=1.5cm,
        font=\sffamily\footnotesize,
        % Styles
        data/.style={
            single arrow, 
            draw=black!60, 
            thick, 
            fill=gray!10, 
            minimum height=1.5cm, 
            shape border rotate=0,
            align=center,
            drop shadow
        },
        barrier/.style={
            rectangle, 
            draw=black!60, 
            very thick, 
            rounded corners=4pt, 
            minimum width=2.2cm, 
            minimum height=2.5cm, 
            align=center,
            drop shadow,
            font=\sffamily\scriptsize\bfseries
        },
        database/.style={
            cylinder, 
            cylinder uses custom fill, 
            cylinder body fill=orange!10, 
            cylinder end fill=orange!30, 
            shape border rotate=90, 
            aspect=0.25, 
            draw=black!60, 
            thick, 
            minimum width=1.5cm, 
            minimum height=1.5cm,
            align=center,
            drop shadow
        },
        reject/.style={
            -{Latex}, 
            thick, 
            draw=red!70, 
            dashed
        },
        label_reject/.style={
            below,
            align=center,
            text=red!70, 
            font=\tiny\itshape
        }
    ]

    % Input Data
    \node[data, fill=white] (input) {
        Conjunto de datos
    };

    % Layer 1: Preprocessing
    \node[barrier, right=0cm of input, fill=green!10] (layer1) {
        CAPA 1 \\[.2cm]
        Preprocesado\\
        en Python \\
        (\texttt{pandas})
    };

    % Layer 2: Database constraints
    \node[barrier, right=0.8cm of layer1, fill=yellow!10] (layer2) {
        CAPA 2 \\[.2cm]
        Semántica en BBDD \\
        (\texttt{CHECK}, \\
        \texttt{NOT NULL}, \\
        \texttt{TRIGGER})
    };

    % Layer 3: Correct implementation of queries
    \node[barrier, right=0.8cm of layer2, fill=red!10] (layer3) {
        CAPA 3 \\[.2cm]
        Programación\\
        defensiva \\
        en queries
    };

    % Arrows
    \draw[-{Latex}, thick] (layer1) -- (layer2);
    \draw[-{Latex}, thick] (layer2) -- (layer3);

    \draw[reject] (layer1.south) -- ++(0,-0.8) node[label_reject] {Campos faltantes};
    \draw[reject] (layer2.south) -- ++(0,-0.8) node[label_reject] {
        Violación semántica \\
        o \\
        tipos incorrectos
    };
    \draw[reject] (layer3.south) -- ++(0,-0.8) node[label_reject] {Resultados incoherentes};

    \end{tikzpicture}
    \caption{\textit{Pipeline} de defensa en profundidad.}
    \label{design:layered_defense}
\end{figure}

\subsection{Inyección de dependencias}

Todo el proceso de pruebas hace uso de \textit{fixtures}, de tal modo que los casos de prueba no crean los objetos que necesitan para su ejecución, sino que es el propio \texttt{pytest} el encargado de instanciarlos y pasarlos como argumentos.

\subsection{Fachada}

El controlador se encarga de encapsular la complejidad inherente a la gestión de conexiones en \texttt{database.py}, proporcionando al resto de scripts una interfaz simplificada para lanzar sus consultas, mediante los objetos \texttt{pg\_query\_runner} y \texttt{neo4j\_query\_runner}.

\section{Datos}

El diseño de la capa de persistencia se caracteriza por la dualidad de modelos. Aunque la fuente de datos es la misma (los ficheros CSV del estándar GTFS), la aproximación al almacenamiento difiere entre el motor relacional y el orientado a grafos.

\subsection{Ciclo de vida de la información}

A nivel de flujo, la información sigue un modelo lineal caracterizado por su simplicidad:

\begin{enumerate}
    \item \textbf{Preprocesado:} El script \texttt{process\_dataset.py} se encarga de limpiar el conjunto de datos, garantizando que no hay columnas faltantes (aunque se admiten valores nulos).
    \item \textbf{Modelado e importación:} Bajo la orquestación de \texttt{import.py}, los motores de bases de datos generan el esquema, leen los ficheros CSV del conjunto de datos y almacenan la información del modo descrito en los siguientes subapartados.
    \item \textbf{Enriquecimiento:} A continuación, tanto PostgreSQL como Neo4J enriquecen la información importada con datos de índole geoespacial, como puntos que representan la ubicación de las paradas o líneas que siguen el trazado de los trayectos.
    \item \textbf{Recuperación:} Mediante los scripts de consulta o evaluación de rendimiento, ubicados en el directorio \texttt{Scripts}, se recupera la información de las bases de datos y se utiliza para producir resultados.
    \item \textbf{Almacenamiento y visualización:} Los resultados generados son visualizados, en el caso de los mapas o las gráficas, o almacenados en ficheros JSON, en el caso de los tiempos de importación y ejecución.
\end{enumerate}

\subsection{Modelo relacional (PostgreSQL)}

El modelado en PostgreSQL consiste en un \textit{mapeo} casi\footnote{Algunos elementos, como las claves primarias con valores potencialmente nulos, no se pueden implementar de forma directa en PostgreSQL, así que se han buscado alternativas: en ese caso, índices que garantizan unicidad.} inmediato (1:1) de las entidades definidas en el estándar GTFS. En la figura \ref{design:er_model} se puede apreciar una simplificación del modelo entidad-relación (MER) seguido, considerando las entidades básicas.

A nivel de implementación, se utiliza un script por cada una de las tablas SQL, ubicados todos ellos en el directorio \texttt{Scripts/PostgreSQL/Import}. También, en lugar de almacenar los datos de tipos enumerados (por ejemplo, estatus de accesibilidad de paradas o direcciones de viaje) como valores enteros, de han generado tablas adicionales para cada uno de ellos.

En lo relativo a los componentes opcionales del estándar GTFS, sus tablas se crean siempre, independientemente de que el conjunto de datos vaya a insertar datos en ellas o no.

\begin{figure}[ht]
    \centering
    \begin{tikzpicture}[
        node distance=2.5cm,
        entity/.style={
            rectangle, 
            draw=black, 
            thick, 
            fill=blue!5, 
            minimum width=2.5cm, 
            minimum height=1cm,
            align=center,
            font=\sffamily\small
        },
        relation/.style={
            diamond, 
            draw=black, 
            thick, 
            fill=gray!10, 
            aspect=2, 
            inner sep=1pt,
            font=\sffamily\tiny
        },
        link/.style={
            -, 
            thick
        }
    ]

    % Entities
    \node[entity] (agency) {Agency};
    \node[entity, right=of agency] (route) {Route};
    \node[entity, below=of route] (trip) {Trip};
    \node[entity, left=of trip] (service) {Service};
    \node[entity, right=of trip] (shape) {Shape\\(LineString)};
    \node[entity, below=of trip] (stoptime) {Stop Time};
    \node[entity, right=of stoptime] (stop) {Stop\\(Point)};

    % Relations
    \draw[link] (agency) -- node[relation] {Opera} (route);
    \draw[link] (route) -- node[relation] {Cuenta con} (trip);
    \draw[link] (trip) -- node[relation] {Sigue} (shape);
    \draw[link] (trip) -- node[relation] {Formado por} (stoptime);
    \draw[link] (stop) -- node[relation] {Ubicado en} (stoptime);
    \draw[link] (service) -- node[relation] {Tiene activos} (trip);

    % Cardinalities
    \node[anchor=south west, font=\tiny] at (agency.east) {1};
    \node[anchor=south east, font=\tiny] at (route.west) {N};
    
    \node[anchor=north west, font=\tiny] at (route.south) {1};
    \node[anchor=south west, font=\tiny] at (trip.north) {N};

    \node[anchor=south west, font=\tiny] at (service.east) {1};
    \node[anchor=south east, font=\tiny] at (trip.west) {N};

    \node[anchor=south west, font=\tiny] at (trip.east) {1};
    \node[anchor=south east, font=\tiny] at (shape.west) {1};

    \node[anchor=north west, font=\tiny] at (trip.south) {1};
    \node[anchor=south west, font=\tiny] at (stoptime.north) {N};

    \node[anchor=south west, font=\tiny] at (stoptime.east) {N};
    \node[anchor=south east, font=\tiny] at (stop.west) {1};

    \end{tikzpicture}
    \caption{Modelo entidad-relación (simplificado) seguido en PostgreSQL.}
    \label{design:er_model}
\end{figure}

\subsection{Modelo orientado a grafos (Neo4J)}

En Neo4J, el almacenamiento de la información abandona la estructura tabular para adoptar una topología de red. La transformación se lleva a cabo de la forma más intuitiva: cada clave foránea en el modelo original se transforma en una arista, es decir, una relación explícita, en el grafo. De esta forma, se obtiene una estructura como la representada en la figura \ref{design:meta_graph}, un meta-grafo simplificado.

Al igual que en el caso de PostgreSQL, la semántica de los campos o propiedades se preserva mediante restricciones de valores, pero en este caso, en lugar de utilizar \texttt{CHECK} en la definición de una tabla, se emplean \textit{triggers} que comprueban los valores durante el proceso de importación.

\begin{figure}[H]
    \centering
    \begin{tikzpicture}[
        node distance=2.5cm,
        node_style/.style={
            circle, 
            draw=black, 
            thick, 
            fill=orange!10, 
            minimum size=1.3cm,
            align=center,
            font=\sffamily\footnotesize\bfseries
        },
        rel_style/.style={
            -{Latex}, 
            thick, 
            draw=black!80
        },
        label_style/.style={
            midway, 
            sloped, 
            above, 
            font=\sffamily\tiny
        }
    ]

    % Nodes
    \node[node_style] (agency) {:Agency};
    \node[node_style, right=of agency] (route) {:Route};
    \node[node_style, below=of route] (trip) {:Trip};
    \node[node_style, left=of trip] (service) {:Service};
    \node[node_style, below=of service] (day) {:Day};
    \node[node_style, right=of trip] (shape) {:Shape};
    \node[node_style, below=of trip] (stoptime) {:StopTime};
    \node[node_style, right=of stoptime] (stop) {:Stop};

    % Relationships
    \draw[rel_style] (route) -- node[label_style] {:OPERATED\_BY} (agency);
    \draw[rel_style] (trip) -- node[label_style] {:FOLLOWS} (route);
    \draw[rel_style] (trip) -- node[label_style] {:SCHEDULED\_BY} (service);
    \draw[rel_style] (trip) -- node[label_style] {:HAS\_SHAPE} (shape);
    \draw[rel_style] (stoptime) -- node[label_style] {:PART\_OF} (trip);
    \draw[rel_style] (stoptime) -- node[label_style] {:LOCATED\_AT} (stop);
    \draw[rel_style, bend right=45] (service) to node[label_style, above, rotate=180] {:STARTS\_ON} (day);
    \draw[rel_style, bend left=45] (service) to node[label_style, below] {:ENDS\_ON} (day);

    \end{tikzpicture}
    \caption{Meta-grafo (simplificado) seguido en Neo4J.}
    \label{design:meta_graph}
\end{figure}

\section{Algoritmos y consultas}

El sistema implementa dos clases de lógica de procesamiento: mientras que la mayoría de consultas son declarativas, y su ejecución es planificada por los motores de base de datos, los algoritmos de alcanzabilidad y búsqueda de caminos más cortos son imperativos, y se implementan como procedimientos.

A causa de ello, y dado que Neo4J no proporciona soporte nativo para código imperativo, dichas consultas no se han implementado en el motor. Sería necesario desarrollar un plugin en Java que interopere con Neo4J, pero esto se considera fuera del ámbito del proyecto.

\subsection{Lógica declarativa}

La amplia mayoría de las consultas predefinidas han sido implementadas de modo puramente declarativo, tanto en PostgreSQL como en Neo4J. En esta sección se presentan algunos de los detalles de implementación más destacables.

\subsubsection{Procesamiento de secuencias}

Una parte fundamental del análisis de datos de transporte requiere operar sobre secuencias ordenadas de datos. Algunos ejemplos de ello son el cálculo de duraciones en los trayectos, la determinación de tiempos de espera en las paradas, o la identificación de segmentos de rutas.

Para satisfacer esta necesidad, se han utilizado dos patrones de consulta, uno por cada motor de base de datos: \textbf{funciones de ventana} en PostgreSQL y \textbf{procesamiento de colecciones} en Neo4J.

En el primer caso, PostgreSQL permite evitar los costosos \textit{self-joins} mediante el uso de funciones de ventana, como \texttt{LEAD}, que permite acceder a valores posteriores en una secuencia ordenada. Además, también se puede acceder a los extremos de la ventana mediante \texttt{FIRST\_VALUE} y \texttt{LAST\_VALUE}. En la figura \ref{design:window_functions} se muestra un ejemplo de su uso para calcular las duraciones de los trayectos.

\begin{figure}[ht]
    \centering
    \begin{huggedminted}{sql}
SELECT DISTINCT
    trip_id,
    EXTRACT(
        EPOCH FROM (
            LAST_VALUE(arrival_time::interval) OVER trip_window -
            FIRST_VALUE(departure_time::interval) OVER trip_window
        )
    ) AS duration_seconds
FROM stop_time
WINDOW trip_window AS (
    PARTITION BY trip_id
    ORDER BY stop_sequence
    ROWS BETWEEN UNBOUNDED PRECEDING AND UNBOUNDED FOLLOWING
)
    \end{huggedminted}
    \caption{Ejemplo de funciones ventana en PostgreSQL.}
    \label{design:window_functions}
\end{figure}

Por otro lado, en el caso de Neo4J, Cypher\cite{cypher} no dispone de un equivalente directo a las funciones de ventana, por lo que se opta por la materialización de secuencias en listas, mediante \texttt{collect}, y su procesamiento con funciones proporcionadas por la biblioteca APOC\cite{apoc}, o con extensiones de lista (\textit{list comprehensions}). En la figura \ref{design:collect} se muestra un ejemplo de su uso para calcular todos los tiempos de espera entre trayectos para cada par (ruta, parada).

\begin{figure}[ht]
    \centering
    \begin{huggedminted}{cypher}
MATCH (r:Route)<-[:FOLLOWS]-(t:Trip)-[:SCHEDULED_BY]->(:Service {id: s_id})
MATCH (s:Stop)<-[:LOCATED_AT]-(stt:StopTime)-[:PART_OF]->(t)

[...]

WITH r, s, direction_id as direction_id, collect(departure_time) as dt
WHERE size(dt) >= 2     // A single departure is not enough.

WITH r, s, direction_id,
     [i IN range(0, size(dt) - 2) | dt[i+1] - dt[i]] as rs_headways
    \end{huggedminted}
    \caption{Ejemplo de procesamiento de colecciones en Neo4J.}
    \label{design:collect}
\end{figure}

\subsubsection{Agregación en conjuntos}

Algunas consultas, como \texttt{daily\_status}, pueden ofrecer un rendimiento muy superior si agrupamos todos los días con los mismos servicios activos en un solo conjunto, dado que se eliminan los cálculos duplicados.

En PostgreSQL, esto se logra mediante la creación de una tabla adicional, llamada \texttt{day\_service\_sets}, cuyas filas contienen tuplas (\textit{service\_dates}, \textit{service\_set}), donde el primer elemento es un conjunto de fechas con los mismos servicios activos, y el segundo, los identificadores de dichos servicios.

Asimismo, para garantizar la eficiencia a la hora de consultar los datos de esta tabla, se crean \textbf{indices GIN}, utilizados junto a los operadores \texttt{\&\&} y \texttt{<@} en la implementación de las consultas.

\subsubsection{Optimización espacial}

Varias de las consultas predefinidas hacen uso de datos espaciales. Por ejemplo, en \texttt{stops\_within\_distance}, es necesario determinar puntos geométricamente cercanos.

El enfoque \textit{naïve} consiste en comparar la distancia con todo el resto de puntos, pero esto resulta ser muy costoso en términos de rendimiento.

Afortunadamente, tanto PostGIS\cite{postgis} como Neo4J-Spatial\cite{neo4j-spatial} proporcionan mecanismos para acelerar estas consultas, como los \textbf{índices GiST}, de los que se ha hecho uso, creándolos inmediatamente después de llevar a cabo la importación de los datos.

\subsubsection{Metaconsultas}

Una de las mayores limitaciones de Neo4J es su incapacidad de definir funciones procedimientos mediante Cypher, requiriendo hacer uso de plugins externos escritos en Java.

Sin embargo, y dado que se busca proporcionar un catálogo de consultas, este problema se ha abordado utilizando un patrón de \textbf{metaconsultas}. Gracias a la biblioteca \texttt{APOC}, podemos ejecutar código Cypher dinámico, con el procedimiento \texttt{apoc.cypher.run}, lo que a su vez nos permite almacenar el texto de las consultas como una propiedad en un nodo etiquetado como \textit{CypherQuery}, recuperarlo posteriormente, y lanzar dicha consulta.

En la figura \ref{design:metaqueries} se puede apreciar la invocación de una de las consultas predefinidas a través de esta solución.

\begin{figure}[h]
    \centering
    \begin{huggedminted}{cypher}
MATCH (cq: CypherQuery {name: 'headway_stats'})
CALL apoc.cypher.run(cq.statement, {curr_date: '2025-12-25'})
YIELD value
RETURN value;
    \end{huggedminted}
    \caption{Código de invocación de una consulta predefinida en Neo4J.}
    \label{design:metaqueries}
\end{figure}

\subsection{Enrutamiento híbrido}

El componente más complejo del sistema a nivel algorítmico es la función de cálculo de alcanzabilidad, denominada \texttt{earliest\_arrivals}, que se utiliza también para hallar el camino más corto entre dos paradas.

Dada la naturaleza del problema, que requiere considerar tanto factores dependientes del tiempo (los trayectos) como independientes (los transbordos y los desplazamientos a pie), no es posible hacer uso ni del algoritmo de Dijkstra\cite{dijkstra}, ni del \textit{Connection Scan Algorithm}\cite{csa} (CSA). Asimismo, tampoco se puede optar por un algoritmo proporcionado por \texttt{pgrouting} o Neo4J, ya que no soportan factores dependientes del tiempo.

Por tanto, como parte del proyecto, se desarrolla e implementa un algoritmo híbrido, que toma el componente fundamental de CSA, iterando sobre las conexiones entre nodos ordenadas temporalmente por hora de salida, y añade una fase de relajación local, similar al algoritmo de Dijkstra.

\subsubsection{Fundamento teórico}

El algoritmo CSA original mantiene un vector de estado $E$, donde $E[p]_t$ representa la hora mínima de llegada a la parada $p$ en el instante $t$. Durante la ejecución del algoritmo, este vector se actualiza a medida que avanza el tiempo. Dado que el coste de desplazamiento nunca es negativo, se puede asegurar que:
\[
    \forall p, t, \enspace E[p]_t \leq t \implies \nexists t' > t: E[p]_{t'} < E[p]_t \qquad \text{(Optimalidad global)}
\]
En la variante híbrida, al confirmar un tiempo de llegada globalmente óptimo $t$, a una parada $p$, se inyectan conexiones de coste estático y tiempo de salida $t$ (representando caminar y cada uno de los posibles transbordos), y se procesan inmediatamente para encontrar tiempos de llegada a los vecinos de $p$ potencialmente mejores a los almacenados en el vector de estado $E$.

Por otra parte, como hemos mencionado antes, CSA proporciona resultados óptimos únicamente si garantizamos que $E[p]_t$ representa la hora mínima de llegada a la parada $p$ en el instante $t$. Dado que las relajaciones de los nodos pueden mejorar el tiempo de llegada a alguno de sus vecinos, es \textbf{imprescindible} que dichas relajaciones se produzcan en cuanto se confirme la optimalidad global de un camino.

\subsubsection{Implementación}

En la figura \ref{design:modified_csa} se muestra una implementación simplificada del algoritmo en PL/pgSQL.

\begin{figure}[p]
    \centering
    \begin{huggedminted}{postgresql}
-- Initialize data structures.
-- Array of results that is progressively filled.
SELECT ... INTO results FROM "stop" s;

-- Priority queue of stops, sorted by earliest arrival time.
CREATE TEMP TABLE pqueue (...) ON COMMIT DROP;
INSERT INTO pqueue SELECT ... FROM "stop" s;

-- Lists of neighbors and transfers for each stop.
neighbors := (SELECT ...); transfers := (SELECT ...);

-- Loop through chronologically sorted connections.
FOR conn IN
    SELECT *
    FROM connections c
    ORDER BY departure_time ASC
LOOP
    -- Iterate over arrival times confirmed as optimal.
    IF last_scanned_departure_time < conn.departure_time THEN
        FOR conf_stop IN
            -- Remove the stops from the queue after processing them.
            DELETE FROM pqueue pq
            WHERE pq.arrival_time <= conn.departure_time
            RETURNING *
        LOOP
            -- Iterate over neighbors.
            FOREACH nei IN ARRAY neighbors[conf_stop.stop_idx].val
            LOOP
                -- Calculate arrival time at the neighbor stop.
                new_arrival := ...

                -- Update arrival time if earlier.
                IF new_arrival < results[nei].earliest_arrival THEN
                    results[nei] = ...;
                    UPDATE pqueue SET ... WHERE ...;
                END IF;
            END LOOP;

            -- Iterate over transfers.
            FOREACH transfer IN ARRAY transfers[conf_stop.stop_idx].val
            LOOP
                -- Analogous to neighbor handling.
            END LOOP;
        END LOOP;
    END IF;

    -- Process the next regular connection (trip).
    -- We check if it's reachable from its departure stop.
    IF results[departure_stop].earliest_arrival <= conn.departure_time THEN
        -- Analogous to the neighbor handling.
    END IF;
END LOOP;
    \end{huggedminted}
    \caption{Implementación simplificada del algoritmo CSA híbrido.}
    \label{design:modified_csa}
\end{figure}







\cleardoublepage

\chapter{Pruebas} \label{chapter:tests}

Una vez construido el sistema, se procede a evaluar su correctitud mediante la ejecución de un plan de pruebas, que sirve también de base para comparativas entre motores, ya en el capítulo \ref{chapter:comparison}.

\section{Estrategia de pruebas}

\subsection{Alcance}

El plan de pruebas busca asegurar el correcto funcionamiento de los siguientes aspectos:

\begin{itemize}
    \item \textbf{Integración de componentes:} Conectividad e interoperabilidad entre los sistemas gestores de bases de datos y el módulo de control, en Python, a través de drivers. Su verificación se lleva a cabo de manera implícita al ejecutar las demás pruebas.
    \item \textbf{Integridad de datos:} Preservación de la información contenida en los conjuntos de datos durante el proceso de importación, incluyendo también la integridad referencial y respetando los tipos de datos.
    \item \textbf{Algoritmos de consulta:} Exactitud de los resultados devueltos por las consultas implementadas. Dado que se utilizan dos sistemas gestores de bases de datos diferentes, el objetivo principal es asegurar que los resultados proporcionados por ambos son iguales, lo que sirve de evidencia sobre su equivalencia semántica.
\end{itemize}

Cabe mencionar que este capítulo se centra de manera exclusiva en la \textbf{correctitud}, es decir, que el sistema funcione de la manera esperada.

\subsection{Herramientas e infraestructura}

La tecnología principal sobre la que se articula el plan de pruebas es \textbf{pytest}. Esta elección responde a su capacidad de llevar a cabo los siguientes aspectos:

\begin{itemize}
    \item Automatización de la ejecución, permitiendo lanzar todas las pruebas con un solo comando.
    \item Reporte detallado de errores, indicando exactamente qué casos de prueba han fallado.
    \item Creación de un entorno controlado que garantice la reproducibilidad de los resultados.
    \item Gestión de elementos comunes a todas las pruebas, como las conexiones a ambas bases de datos, haciendo uso de \textit{fixtures}.
\end{itemize}

En cuanto a otras bibliotecas de Python, se emplean \textbf{psycopg} y \textbf{neo4j-driver} para las conexiones con las bases de datos, y \textbf{hypothesis} para las pruebas basadas en propiedades.

Al igual que durante la operación habitual del sistema, las bases de datos se ejecutan como contenedores de \textbf{Docker}, lo que garantiza aislamiento del sistema operativo y de sus bibliotecas.

\subsection{Conjuntos de datos}

A fin de comprobar la robustez del sistema frente a diferentes estructuras de red, convenciones en cuanto a la nomenclatura, y volúmenes de datos, se ha elegido un corpus que engloba 10 sistemas de transporte en varios continentes, detallado en el cuadro \ref{test:datasets}.

Llevar a cabo las pruebas utilizando estos conjuntos de datos permite asegurar que el sistema funciona correctamente bajo condiciones de internacionalización (codificación UTF-8 para caracteres cirílicos e ideogramas chinos) e independientemente de la topología de red (soportando desde redes de metro densas hasta líneas de autobús regionales).

\begin{table}[H]
    \centering
    \renewcommand{\arraystretch}{1.2}
    \begin{tabular}{@{}llc@{}}
        \toprule
        \textbf{Nombre} & \textbf{Tipo} & \textbf{Tamaño (.txt)} \\
        \midrule
        \mbox{\href{https://mobilitydatabase.org/feeds/gtfs/mdb-793}{Madrid (EMT)}} &
        Urbano (bus) &
        123 MB \\
        \addlinespace
        \mbox{\href{https://mobilitydatabase.org/feeds/gtfs/mdb-767}{Praga}} &
        Regional mixto (bus, metro, tren) &
        255 MB \\
        \addlinespace
        \mbox{\href{https://mobilitydatabase.org/feeds/gtfs/mdb-1924}{Hong Kong}} &
        \makecell[lt]{
            Urbano mixto (bus, ferry, funicular) \\
            Codificación UTF-8 (caracteres chinos) \\
            Alta densidad
        } &
        113 MB \\
        \addlinespace
        \mbox{\href{https://mobilitydatabase.org/feeds/gtfs/mdb-2012}{Bogotá (SIMUR)}} &
        \makecell[lt]{
            Urbano (bus) \\
            Gran número de trayectos ($>$ 150.000)
        } &
        594 MB \\
        \addlinespace
        \mbox{\href{https://mobilitydatabase.org/feeds/gtfs/mdb-1076}{Singapur}} &
        \makecell[lt]{
            Urbano mixto (bus, metro) \\
            Modelo basado en frecuencias \\
            Alta densidad
        } &
        8 MB \\
        \addlinespace
        \mbox{\href{https://mobilitydatabase.org/feeds/gtfs/mdb-2821}{Galicia (Xunta)}} &
        \makecell[lt]{
            Regional (bus) \\
            Contexto conocido \\
            Gran número de rutas ($>$ 6.500) \\
            Gran número de paradas ($>$ 25.000)
        } &
        498 MB \\
        \addlinespace
        \mbox{\href{https://mobilitydatabase.org/feeds/gtfs/mdb-2333}{Munich}} &
        \makecell[lt]{
            Urbano mixto (bus, tranvía)
        } &
        140 MB \\
        \addlinespace
        \mbox{\href{https://busmaps.com/en/russia/open-data-portal-moscow/moscow-official}{Moscú}} &
        \makecell[lt]{
            Regional (bus) \\
            Codificación UTF-8 (caracteres cirílicos)
        } &
        196 MB \\
        \addlinespace
        \mbox{\href{https://www.mta.info/developers}{Nueva York (MTA)}} &
        \makecell[lt]{
            Urbano (metro) \\
            Referencia en el sector
        } &
        49 MB \\
        \addlinespace
        \mbox{\href{https://mobilitydatabase.org/feeds/gtfs/mdb-1027}{Belgrado}} &
        \makecell[lt]{
            Urbano mixto (bus, tranvía) \\
        } &
        129 MB \\
        \bottomrule
    \end{tabular}
    \caption{Conjuntos de datos utilizados para las pruebas.}
    \label{test:datasets}
\end{table}

\subsection{Metodología}

Dado el volumen de los conjuntos de datos elegidos, la estrategia de pruebas busca minimizar la necesidad de definir resultados esperados de manera manual, ya que resulta inabordable. En su lugar, se opta por los siguientes enfoques, complementarios entre sí:

\begin{itemize}
    \item \textbf{Validación cruzada:} Aprovechando la redundancia entre ambos sistemas gestores de bases de datos, PostgreSQL y Neo4J, se verifica que los resultados generados por estos coinciden entre sí, tanto para comprobar la \textit{integridad} de los datos como la \textit{consistencia} en la implementación de las consultas.
    \item \textbf{Análisis de casos límite:} Se evalúa la \textit{robustez} del sistema ante valores de entrada atípicos o extremos, garantizando que no se producen errores al forzar la lógica de implementación de las consultas.
    \item \textbf{Comprobación basada en propiedades:} Con el objetivo de asegurar la \textit{plausibilidad} de los resultados de las consultas, se definen propiedades matemáticas que estos han de cumplir, así como criterios de coherencia basados en el dominio del problema.
\end{itemize}

\section{Integridad de los datos}

A la hora de comprobar que el proceso de importación se ha realizado con éxito, el enfoque clave es la validación cruzada entre ambos motores de bases de datos. Para llevarla a cabo:

\begin{enumerate}
    \item Se recupera la información que acaba de ser importada mediante consultas, con sus respectivos \texttt{JOIN} en el caso de PostgreSQL y \textit{pattern matching} en el de Neo4J.
    \item Se ordenan los resultados ya en Python, para evitar problemas con las diferencias en el orden proporcionado por las BBDD a causa de la \textit{collation}.
    \item Se realiza una comparación extensiva, que asegure tanto que el número de resultados es el mismo, como que son cualitativamente idénticos.
\end{enumerate}

Se puede ver un modelo de implementación de estas pruebas en la figura \ref{test:integrity_code}, y una muestra de ejecución, sobre el conjunto de datos de Galicia, en la figura \ref{test:integrity}.

\begin{figure}[p]
    \centering
    \begin{huggedminted}{python}
def test_area_data_consistency(pg_query_runner, neo4j_query_runner):
    pg_data = pg_query_runner(
        "SELECT * FROM area ORDER BY area_id;", ())
    neo4j_data = neo4j_query_runner(
        "MATCH (a:Area) RETURN a ORDER BY a.id;", {})
    
    pg_count, neo4j_count = len(pg_data), len(neo4j_data)

    assert pg_count == neo4j_count

    def to_canonical_tuple(row):
        return (row.get('area_id') or row.get('id'),
                row.get('area_name') or row.get('name'))

    pg_tuples = [to_canonical_tuple(row) for row in pg_data]
    neo4j_tuples = [to_canonical_tuple(row) for row in neo4j_data]

    pg_tuples.sort()
    neo4j_tuples.sort()

    assert pg_tuples == neo4j_tuples
    \end{huggedminted}
    \caption{Muestra de código de las pruebas de importación (simplificada).}
    \label{test:integrity_code}
\end{figure}

\begin{figure}[p]
    \centering
    \includegraphics[width=0.9\linewidth]{figuras/test_import.png}
    \caption{Ejecución de las pruebas de importación (\textit{dataset} de Galicia).}
    \label{test:integrity}
\end{figure}

\section{Algoritmos de consulta}

A diferencia de las pruebas de importación, para asegurar el correcto funcionamiento de las consultas, debemos elegir un amplio espectro de casos de prueba, que evalúen su desempeño tanto en \textbf{condiciones habituales} como en \textbf{situaciones atípicas}. Al mismo tiempo, es conveniente comprobar que se mantienen ciertos \textbf{invariantes} en los resultados.

Los detalles de implementación se tratan en las siguientes subsecciones, y en la figura \ref{test:algorithms} se proporciona un ejemplo de ejecución.

\subsection{Casos de prueba aleatorizados} \label{test:random_queries}

Para simular consultas que puedan ser realizadas por un usuario final, se implementan casos de prueba generados de manera aleatoria. Sin embargo, estos no son puramente estocásticos, a fin de evitar lanzar numerosas consultas triviales, cuyos resultados siempre sean nulos:

\begin{enumerate}
    \item \textbf{Descubrimiento:} Se consulta la base de datos para determinar el rango de valores válidos, tales como el rango de fechas de servicio, o la \textit{bounding box} del conjunto de paradas.
    \item \textbf{Selección:} En base a los rangos obtenidos en el paso anterior, se eligen argumentos válidos para crear un caso de prueba.
    \item \textbf{Consulta y comprobación}: Se lanza el caso de prueba y se verifica que los resultados proporcionados por ambas bases de datos son idénticos.
\end{enumerate}

En cuanto al número de casos de prueba, para cada conjunto de datos, se han considerado 30 para la mayoría de consultas, y 10 para aquellas que requieren un mayor tiempo de ejecución.

\subsection{Valores límite}

Aunque, en la mayoría de situaciones, un usuario final no vaya a realizar consultas con valores que no tengan sentido en el dominio del problema, es importante tener en cuenta dichos casos límite para garantizar la robustez de los algoritmos usados. Algunos ejemplos de argumentos considerados son:

\begin{itemize}
    \item Fechas inmediatamente anteriores al comienzo del servicio o posteriores al fin de este.
    \item Coordenadas de lugares aislados como los polos norte y sur.
    \item Distancias de búsqueda nulas.
    \item Paradas inexistentes.
    \item Rutas inexistentes.
    \item Horas del día tardías.
    \item Intervalos temporales muy cortos.
\end{itemize}

Adicionalmente, la figura \ref{test:edge_cases_code} presenta un ejemplo de implementación de estas pruebas, incluyendo varias de las clases de argumentos listados.

\begin{figure}[h]
    \centering
    \begin{huggedminted}{python}
# Edge cases for stops_within_distance.
def test_edge_cases(pg_query_runner, neo4j_query_runner, bounding_box):
    # North Pole.
    results = run_test_case(origin_lat=90.0, origin_lon=0.0,
                            seek_dist=MAX_SEARCH_DISTANCE_METERS)
    assert len(results) == 0

    # South Pole.
    results = run_test_case(origin_lat=-90.0, origin_lon=0.0,
                            seek_dist=MAX_SEARCH_DISTANCE_METERS)
    assert len(results) == 0

    # Null search distance (multiple iterations).
    for i in range(RANDOM_TEST_COUNT):
        lat, lon = random_point_in_bbox(bounding_box)
        results = run_test_case(origin_lat=lat, origin_lon=lon, seek_dist=0)
        assert len(results) == 0

    # Well outside the bounding box, with long range.
    outside_lat = bounding_box['max_lat'] + 10.0
    outside_lon = bounding_box['max_lon'] + 10.0
    distance = 5000
    results = run_test_case(origin_lat=outside_lat, origin_lon=outside_lon,
                            seek_dist=distance)
    assert len(results) == 0
    \end{huggedminted}
    \caption{Muestra de código de las pruebas de valores límite (simplificada).}
    \label{test:edge_cases_code}
\end{figure}

\subsection{Invariantes}

Por último, utilizando la biblioteca \textit{hypothesis}\cite{hypothesis}, se implementan pruebas que tienen como objetivo asegurar el cumplimiento de ciertas propiedades en los resultados de las consultas.

A modo de ejemplo, los resultados de una consulta sobre un rango temporal han de estar contenidos en los resultados de otra consulta sobre otro rango temporal mayor, es decir:

\[
    \forall a, a', b, b' \in [d_{begin}, d_{end}],  a \le a' \le b' \le b \implies q(a', b') \subseteq q(a, b)
\]

La implementación de la prueba correspondiente puede verse en la figura \ref{test:invariants_code}.

Asimismo, durante la ejecución de los casos de prueba aleatorizados, se introducen \textit{comprobaciones de plausibilidad}, como que el número de elementos en listas no es negativo, que los resultados se encuentran dentro del rango de búsqueda, o que estos están ordenados de la manera esperada.

\begin{figure}[p]
    \centering
    \begin{huggedminted}{python}
# Property: Subrange queries should return the same results as
#           the full range query filtered to that subrange.
@given(
    sub_start=st.dates(
        min_value=service_date_range['min_date'],
        max_value=service_date_range['max_date']
    ),
    sub_end=st.dates(
        min_value=service_date_range['min_date'],
        max_value=service_date_range['max_date']
    )
)
@settings(deadline=None)
def test_pbt_subrange_consistency(full_pg, full_neo, sub_start, sub_end):
    assume(sub_start <= sub_end)
    
    # Run subrange queries
    sub_pg = pg_query_runner(SQL, (sub_start, sub_end))
    sub_neo = neo4j_query_runner(CYPHER,
              {'start_date': str(sub_start), 'end_date': str(sub_end)})

    # Filter the full-range results to the subrange
    filtered_pg = [r for r in full_pg
                   if sub_start <= r['service_date'] <= sub_end]
    filtered_neo = [r for r in full_neo
                    if sub_start <= r['value']['service_date'] <= sub_end]

    # Check if the property holds.
    assert sub_pg == filtered_pg
    assert sub_neo == filtered_neo
    \end{huggedminted}
    \caption{Muestra de código de las pruebas de invariantes (simplificada).}
    \label{test:invariants_code}
\end{figure}

\begin{figure}[p]
    \centering
    \includegraphics[width=0.9\linewidth]{figuras/test_algorithms.png}
    \caption{Ejecución de las pruebas de consultas (\textit{dataset} de Galicia).}
    \label{test:algorithms}
\end{figure}
\cleardoublepage

\chapter{Evaluación comparativa} \label{chapter:comparison}

Una vez llevado a cabo el proceso de pruebas, y garantizado el correcto funcionamiento del sistema, este capítulo aborda las diferencias entre las implementaciones objeto-relacional (PostgreSQL) y orientada a grafos (Neo4J).

El análisis se centra en tres dimensiones: rendimiento (tiempos de ejecución del proceso de importación y de las consultas), facilidad de uso (modelado y expresión de consultas) y soporte de bibliotecas.

\section{Rendimiento}

Durante el proceso de pruebas, se llevan a cabo mediciones de los tiempos de ejecución, utilizadas en esta comparativa, para todos los conjuntos de datos descritos en la figura \ref{test:datasets}. En el caso de las consultas, se han medido sus tiempos de ejecución en los casos de prueba aleatorizados (sección \ref{test:random_queries}).



\subsection{Tiempo de importación (input)}

El proceso de ingesta de datos muestra las mayores diferencias en términos de rendimiento, como se puede comprobar en la figura \ref{comparison:import_times}. En general, PostgreSQL es \textbf{un orden de magnitud} más rápido que Neo4J, siendo el \textit{speedup} alcanzado de casi 13x.

A este respecto, conviene mencionar que los procesos de importación en ambos sistemas gestores de bases de datos son fundamentalmente diferentes: mientras que PostgreSQL copia los datos directamente de los ficheros .csv, y a continuación comprueba la integridad referencial y que los datos sean válidos en el dominio, Neo4J también ha de construir la topología de la red, lo que implica determinar los nodos origen y destino y almacenar la relación entre ellos.

Por otra parte, cabe mencionar que Neo4J no proporciona soporte\footnote{Existe una herramienta externa, neo4j-admin, que permite cargar ficheros .csv directamente, pero estos deben ser adaptados a un formato específico. Es posible que el tiempo de importación se pueda reducir de esta manera, pero ello complica el procesamiento de datos.} para copiar directamente datos desde un archivo .csv lo que obliga a leer las líneas de manera secuencial, agruparlas en conjuntos de miles de líneas (para evitar desbordamientos de memoria), y crear los nodos correspondientes.

\begin{figure}[ht]
    \centering
    \includegraphics[width=\linewidth]{figuras/import_times.png}
    \caption{Tiempos de importación en PostgreSQL y Neo4J.}
    \label{comparison:import_times}
\end{figure}

\subsection{Tiempo de consulta (output)} \label{comparison:query_section}

En el ámbito de la recuperación y análisis de datos, las diferencias de rendimiento son menores, y, además, dependen de la consulta realizada, como se puede apreciar en la figura \ref{comparison:query_times}. A causa de ello, no tiene sentido considerar el speedup global, dado que no hay un sistema que rinda mejor de manera clara.

Resulta interesante destacar que Neo4J alcanza un mejor rendimiento en aquellas consultas que requieren atravesar múltiples niveles de relaciones desde un número reducido de nodos iniciales, siendo ejemplos \texttt{routes\_by\_speed}, que calcula la velocidad media de una ruta considerando el promedio de sus trayectos, y \texttt{next\_departures}, que determina las próximas salidas desde una parada, lo que implica recorrer la topología para examinar todos los trayectos que parten de ella.

A su vez, PostgreSQL proporciona menores tiempos de ejecución en consultas que involucran cantidades masivas de datos, como \texttt{overlapping\_segments}, que necesita considerar cada uno de los tiempos de llegada de cada trayecto, gracias a que realiza un único \texttt{JOIN} secuencial entre las tablas, y no millones de accesos aleatorios a memoria.

\begin{figure}[ht]
    \centering
    \includegraphics[width=\linewidth]{figuras/query_times.png}
    \caption{Tiempos de consulta en PostgreSQL y Neo4J.}
    \label{comparison:query_times}
\end{figure}



\section{Modelado y expresión}

En ocasiones, resulta preferible potenciar la productividad de los desarrolladores incluso en detrimento del rendimiento que el sistema final pueda ofrecer. Por ello, otro de los aspectos a considerar en esta comparativa es cómo de sencillo resulta tanto modelar la información del sistema como expresar las consultas.

En cuanto al primer aspecto, y dado que la amplia mayoría de la información que debe almacenar la base de datos viene dada por el estándar GTFS, PostgreSQL permite un modelado más rápido y sencillo, dado que los campos en la \href{https://gtfs.org/documentation/schedule/reference/}{especificación} están definidos siguiendo el paradigma de las bases de datos relacionales.

Como muestra la figura \ref{comparison:modeling}, PostgreSQL permite almacenar de manera directa la información tal y como se define en dicha especificación, mientras que Neo4J requiere adaptar el modelo y decidir, por ejemplo, si transformar cada clave foránea de la especificación en una relación entre nodos, e incluso en el caso de proceder de ese modo por simplicidad, la ausencia de valores por defecto obliga a introducir lógica condicional.

\begin{figure}[ht]
    \centering
    \begin{minipage}[t]{0.49\linewidth}
        \vspace{0px}
        \begin{huggedminted}[fontsize=\scriptsize]{sql}
CREATE TABLE trip (
    route_id TEXT NOT NULL
        REFERENCES route(route_id),
    service_id TEXT NOT NULL,
    trip_id TEXT PRIMARY KEY,
    trip_headsign TEXT,
    trip_short_name TEXT,
    direction_id INTEGER
        REFERENCES travel_direction(id),
    block_id TEXT,
    shape_id TEXT,
    wheelchair_accessible INTEGER
        DEFAULT 0
        REFERENCES wheelchair_status(id),
    bikes_allowed INTEGER
        DEFAULT 0
        REFERENCES bicycle_status(id)
);
        \end{huggedminted}
    \end{minipage}%
    \begin{minipage}[t]{0.49\linewidth}
        \vspace{0px}
        \begin{huggedminted}[fontsize=\scriptsize]{cypher}
MATCH (r: Route { id: row.route_id })
MATCH (s: Service { id: row.service_id })
WITH row, r, s
CREATE (t: Trip {
    id: row.trip_id,
    headsign: row.trip_headsign,
    short_name: row.trip_short_name
})
CREATE (t)-[fol: FOLLOWS]->(r)
CREATE (t)-[sch: SCHEDULED_BY]->(s)
CREATE (t)-[hws: HAS_WHEELCHAIR_STATUS]->(ws)
CREATE (t)-[hbs: HAS_BICYCLE_STATUS]->(bs)
WITH row, t
CALL apoc.do.when(
    row.direction_id IS NOT NULL,
    'CREATE (t)-[htd: HAS_TRAVEL_DIR]->(td)
    RETURN row, t',
    'RETURN row, t',
    { row: row, t: t }
) YIELD value
WITH value.row as row, value.t as t
CALL apoc.do.when(
    row.block_id IS NOT NULL,
    'CREATE (t)-[itb: IN_TRIP_BLOCK]->(tb)
    RETURN row, t',
    'RETURN row, t',
    { row: row, t: t }
) YIELD value
        \end{huggedminted}
    \end{minipage}
    \caption{Comparativa de modelado entre PostgreSQL (izq.) y Neo4J (der.).}
    \label{comparison:modeling}
\end{figure}

Sin embargo, Neo4J ofrece una sintaxis que se adecúa mejor a la expresión de múltiples consultas, como es el caso de \texttt{overlapping\_segments} (figura \ref{comparison:querying_neo4j}). Si comparamos esta implementación con la de PostgreSQL (figura \ref{comparison:querying_postgres}), vemos que Neo4J requiere menos código y este resulta más fácil de entender, dado que su lenguaje está específicamente diseñado para aplicar múltiples transformaciones consecutivas a los datos (\textit{pipelining}).

\begin{figure}[p]
    \centering
    \begin{huggedminted}{sql}
WITH trip_segments AS (
    SELECT
        trip_id,
        stop_id AS from_stop_id,
        LEAD(stop_id, 1) OVER
            (PARTITION BY trip_id ORDER BY stop_sequence) AS to_stop_id
    FROM stop_time
),
segment_routes AS (
    SELECT DISTINCT
        ts.from_stop_id,
        ts.to_stop_id,
        t.route_id
    FROM trip_segments AS ts
        JOIN trip AS t ON ts.trip_id = t.trip_id
    WHERE ts.to_stop_id IS NOT NULL
)
SELECT
    s1.stop_name AS from_stop,
    s2.stop_name AS to_stop,
    COUNT(sr.route_id) AS route_count,
    array_agg(r.route_short_name ORDER BY r.route_short_name) AS routes
FROM segment_routes AS sr
    JOIN stop AS s1 ON sr.from_stop_id = s1.stop_id
    JOIN stop AS s2 ON sr.to_stop_id = s2.stop_id
    JOIN route AS r ON sr.route_id = r.route_id
GROUP BY sr.from_stop_id, sr.to_stop_id, s1.stop_name, s2.stop_name
ORDER BY route_count DESC, from_stop, to_stop;
    \end{huggedminted}
    \caption{Código de \texttt{overlapping\_segments} en PostgreSQL.}
    \label{comparison:querying_postgres}
\end{figure}

\begin{figure}[p]
    \centering
    \begin{huggedminted}{cypher}
MATCH (s:Stop)<-[:LOCATED_AT]-(st:StopTime)-[:PART_OF]->(t:Trip)
WITH t, s, st
ORDER BY st.stop_sequence

WITH t, apoc.coll.pairsMin(collect(s)) as stop_sequences
MATCH (t:Trip)-[:FOLLOWS]->(r:Route)
UNWIND stop_sequences as stop_pair
WITH DISTINCT r, stop_pair[0] as from_stop, stop_pair[1] as to_stop

WITH from_stop, to_stop, apoc.coll.sort(collect(r.short_name)) AS routes
ORDER BY size(routes) DESC, from_stop.name, to_stop.name
RETURN from_stop.name AS from_stop,
       to_stop.name AS to_stop,
       size(routes) AS route_count,
       routes
    \end{huggedminted}
    \caption{Código de \texttt{overlapping\_segments} en Neo4J.}
    \label{comparison:querying_neo4j}
\end{figure}

\clearpage % Force query examples to be shown before the next section



\section{Soporte de bibliotecas}

Otro de los aspectos clave a considerar a nivel de productividad del programador es el soporte de bibliotecas, dado que estas simplifican en gran medida la implementación de consultas complejas.

En el caso de PostgreSQL, su extensión PostGIS es un estándar en la industria y ofrece una gran cantidad de funcionalidades, desde las más básicas, como la creación de puntos y líneas, hasta otras más avanzadas como el cálculo de elementos geométricos, como la envoltura cóncava y convexa (\textit{concave/convex hull}) y la generación de diagramas de Voronoi, que permiten derivar áreas de servicio a partir de un conjunto de paradas, o la segmentación de mapas en regiones hexagonales para generar mapas de calor.

En cambio, Neo4J Spatial es una biblioteca más simple y con funcionalidades muy restringidas, ofreciendo, en el contexto de una red de transportes, poco más que creación de puntos y líneas, y cálculo de distancias. Precisamente a causa de ello, se considera inviable la implementación de ciertas consultas, como \texttt{stop\_density\_heatmap}, en Cypher, y no se lleva a cabo.



\section{Resultados}

A modo de conclusión del capítulo, el cuadro \ref{comparison:results} resume los puntos fuertes y débiles identificados en cada uno de los motores de base de datos.

\vspace{1em}

\begin{table}[H]
    \centering
    \renewcommand{\arraystretch}{1.3}
    \begin{tabular}{@{}llp{8.5cm}@{}}
        \toprule
        \textbf{Dimensión} & \textbf{Vencedor} & \textbf{Observaciones} \\
        \midrule
        \textbf{Importación} & PostgreSQL & Alrededor de un orden de magnitud más rápido. \\
        \textbf{Consulta} & — & PostgreSQL rinde mejor en agregación de cantidades masivas de datos, Neo4J en travesía desde un número limitado de nodos iniciales. \\
        \textbf{Modelado} & PostgreSQL & GTFS se define siguiendo el modelo relacional. \\
        \textbf{Expresividad} & Neo4J & Cypher resulta más simple y fácil de entender. \\
        \textbf{Bibliotecas} & PostgreSQL & PostGIS soporta más algoritmos geométricos. \\
        \bottomrule
    \end{tabular}
    \caption{Resumen comparativo: PostgreSQL frente a Neo4J.}
    \label{comparison:results}
\end{table}

\cleardoublepage

\chapter{Conclusiones y posibles ampliaciones} \label{chapter:conclusion}

Este capítulo final sintetiza los resultados obtenidos durante el desarrollo del proyecto, y propone líneas de investigación para futuras mejoras.

\section{Hallazgos}

Este trabajo ha permitido contrastar dos paradigmas de bases de datos diferentes, aplicados a un mismo dominio: la gestión y análisis de redes de transporte público.

La hipótesis inicial sugería que, dada la naturaleza topológica de una red de transporte, que cuenta con nodos y conexiones, una base de datos orientada a grafos como Neo4J debería ofrecer un rendimiento superior. Sin embargo, los resultados empíricos obtenidos matizan esta premisa, dado que:

\begin{enumerate}
    \item PostgreSQL ha ofrecido un mejor rendimiento en tareas de ingesta de datos, particularmente a nivel de importación, siendo un orden de magnitud más rápido que Neo4J.
    \item El tiempo de consulta varía en función de la naturaleza de esta, siendo Neo4J más rápido a la hora de ejecutar consultas que requieren atravesar múltiples niveles de relaciones desde un número reducido de nodos iniciales, mientras que PostgreSQL vence en consultas que requieren agregar cantidades masivas de datos.
    \item PostgreSQL (y todas las bases de datos relacionales, en general) simplifica el proceso de modelado, al estar el estándar GTFS definido siguiendo el modelo relacional.
    \item Neo4J permite expresar las consultas de manera más sencilla y breve, lo que facilita su mantenimiento y comprensión.
    \item El soporte de bibliotecas, fundamental en este dominio, es excelente en PostgreSQL, y bastante más limitado en Neo4J, lo que ha impedido la implementación de ciertas consultas (véase \ref{conclusion:objectives}).
\end{enumerate}

En definitiva, Neo4J es un sistema sencillo de utilizar y flexible, pero que fuera de consultas con requisitos muy específicos, ofrece un rendimiento peor que PostgreSQL. Al mismo tiempo, y aunque una de las ventajas de Neo4J es la mejora en productividad que ofrece al desarrollador a la hora de escribir consultas, esta se ve opacada por el reducido soporte geoespacial de Neo4J Spatial, especialmente en comparación con PostGIS.

\section{Grado de cumplimiento de los objetivos} \label{conclusion:objectives}

A grandes rasgos, se considera que los objetivos principales del proyecto se han alcanzado de manera satisfactoria:

\begin{itemize}
    \item Desarrollo de un sistema capaz de importar y validar datos que siguen el estándar GTFS, abarcando desde redes de transporte locales hasta regionales.
    \item Implementación de un catálogo de consultas predefinidas que el usuario puede lanzar proporcionando únicamente sus argumentos.
    \item Visualización de los resultados tanto mediante integraciones con Python, utilizando Folium, como con herramientas GIS de escritorio como QGIS.
    \item Realización de una comparativa de las ventajas que ofrece cada uno de los motores de base de datos en el contexto de este dominio.
\end{itemize}

Sin embargo, cabe mencionar que debido a ciertas limitaciones en el soporte de bibliotecas de Neo4J, no se han podido implementar algunas consultas, como \texttt{stop\_density\_heatmap}, que requiere la segmentación del mapa en regiones. Asimismo, dado que resulta imposible expresar un algoritmo imperativo en Cypher, no se ha podido implementar \texttt{earliest\_arrivals}, y sería necesario crear un plugin externo, en Java, para ello.

\section{Posibles vías de mejora}

\subsection{Soporte para GTFS Realtime}

Una de las evoluciones naturales del sistema consistiría en añadir soporte para datos en tiempo real, siguiendo el estándar GTFS Realtime\cite{gtfs-rt} (GTFS-RT). Actualmente, el sistema considera una planificación estática, siguiendo horarios teóricos, lo que limita su utilidad en ciertos contextos, como, por ejemplo, frente a retrasos a causa de atascos o incidentes.

Además, la integración de flujos de datos en vivo plantearía nuevos desafíos arquitectónicos, principalmente derivados de la alta frecuencia de escritura, o la actualización de los resultados proporcionados por algoritmos de enrutamiento.

\subsection{Análisis de topología y resiliencia de la red}

Aunque en este trabajo se han abordado ciertos aspectos del análisis basado en topología, como la identificación de paradas consecutivas servidas por un número elevado de rutas (\texttt{overlapping\_segments}), este análisis podría extenderse a otras áreas, como por ejemplo:

\begin{itemize}
    \item Detección de nodos críticos mediante algoritmos de centralidad, como \textit{Betweenness Centrality}\cite{betweenness}, para encontrar cuellos de botella.
    \item Simulación de fallos en la red mediante la eliminación de nodos o aristas para evaluar cómo se degrada la conectividad y determinar la robustez ante incidentes.
    \item Identificación de comunidades mediante algoritmos de \textit{clustering} para hallar zonas fuertemente conectadas entre sí pero aisladas del resto.
\end{itemize}

\subsection{Enrutamiento con restricciones multicriterio}

Otra ampliación relevante sería la implementación de algoritmos de enrutamiento que consideren restricciones complejas por parte del usuario, más allá de minimizar el tiempo de viaje. El ejemplo más claro es la introducción de límites en la distancia recorrida a pie (llegar al destino caminando menos de X metros).

Desde el punto de vista algorítmico, este cambio transformaría el problema en un \textit{Resource Constrained Shortest Path Problem} (RCSPP), que, de manera general, pertenece a la clase de complejidad NP-hard, por lo que sería necesario hacer uso de aproximaciones heurísticas que, potencialmente, producirían resultados subóptimos pero \textit{suficientemente buenos}.


\cleardoublepage

% Aquí empezan os apéndices
\appendix
\cleardoublepage

\chapter{Manual técnico} \label{appendix:development}

\section{Código fuente}

Todo el código fuente del proyecto se encuentra disponible en el \href{https://github.com/rodrada/TFG}{repositorio de GitHub} asociado.
\todo{Add PDFs/PostScript documents with source code.}

El proyecto se compone mayormente de tres partes: scripts en Python (para adaptar los datasets, lanzar el proceso de importación, generar gráficas y mapas con los resultados de las consultas, y ejecutar pruebas), código SQL (para generar las tablas en PostgreSQL, cargar datos y crear las consultas predefinidas), y código Cypher (con el mismo objetivo que el SQL, pero en Neo4J).

\section{Requisitos, dependencias e instalación}

Los requisitos principales del sistema y las dependencias se encuentran descritos en las secciones \ref{user:specifications} y \ref{user:dependencies} del manual de usuario. A mayores de lo necesario para instalar y ejecutar el software, se recomienda contar con un editor de texto o entorno de desarrollo con soporte para Python, SQL y Cypher, como Visual Studio Code.

La instalación, incluyendo el proceso de clonar el repositorio de GitHub,  descargar las bibliotecas de Python necesarias e importar un dataset, se trata en el apartado \ref{user:install} del manual de usuario. Dado que el código está formado únicamente por scripts y no necesita compilación, el proceso de instalación para un desarrollador y un usuario es el mismo.

\clearpage

\section{Estructura del proyecto}

Los directorios del proyecto se disponen de la siguiente manera:

\vspace{0.5em}

\dirtree{%
.1 TFG.
.2 Datasets.
.3 GTFS \DTcomment{Conjuntos de datos, cada uno en un subdirectorio}.
.2 Examples \DTcomment{Imágenes y mapas interactivos (HTML) de muestra}.
.2 Scripts \DTcomment{Scripts para importación y visualización}.
.3 Neo4J.
.4 Import \DTcomment{Carga de datos}.
.5 calendar.cypher.
.5 routes.cypher.
.5 stops.cypher.
.5 ....
.4 launch.sh \DTcomment{Lanzamiento de Neo4J sobre Docker}.
.4 queries.cypher \DTcomment{Definición de consultas}.
.3 PostgreSQL.
.4 Import \DTcomment{Carga de datos}.
.5 calendar.sql.
.5 routes.sql.
.5 stops.sql.
.5 ....
.4 launch.sh \DTcomment{Lanzamiento de PostgreSQL sobre Docker}.
.4 queries.sql \DTcomment{Definición de consultas}.
.3 database.py \DTcomment{Conexiones a las BBDD desde Python}.
.3 departure\_sign\_animation.py.
.3 headway\_stats\_graph.py.
.3 import.py \DTcomment{Script de importación para el usuario}.
.3 process\_dataset.py \DTcomment{Adaptación de datasets}.
.3 route\_speed\_interactive\_map.py.
.3 shortest\_path\_interactive\_map.py.
.3 ....
.2 Tests \DTcomment{Scripts de pruebas}.
.3 Import \DTcomment{Pruebas de importación}.
.3 active\_services.py.
.3 conftest.py \DTcomment{Utilidades para las pruebas}.
.3 daily\_status.py.
.3 departure\_times.py.
.3 ....
.2 Volumes \DTcomment{Datos de Postgres y Neo4J, incluyendo plugins}.
.2 LICENSE \DTcomment{Licencia del código (GPLv3)}.
.2 requirements.txt \DTcomment{Lista de dependencias para \texttt{pip}}.
}

\clearpage

\section{Modificaciones}

\subsection{Actualizaciones en el esquema}

El esquema de ambas bases de datos se define en el código de importación, presente en \verb|Scripts/PostgreSQL/Import| y \verb|Scripts/Neo4J/Import| respectivamente. Cada uno de los archivos del estándar GTFS cuenta con su propio script, nombrado de la misma forma, pero con la extensión correspondiente. Por ejemplo, para \verb|agency.txt|, tenemos los scripts \verb|agency.sql| y \verb|agency.cypher|.

Si únicamente queremos introducir cambios relativos a entidades que ya existen en las bases de datos, es suficiente con modificar los scripts mencionados, pero si la intención es añadir soporte para archivos adicionales, debemos:

\begin{enumerate}
    \item Crear nuestro propio script y nombrarlo siguiendo la convención previa.
    \item Escribir el código siguiendo la estructura los demás scripts de importación.
    \item Modificar el fichero \verb|Scripts/import.py|, incluyendo el nombre del nuevo script en la variable \code{file\_list}. Cabe mencionar que el orden de ejecución de los scripts es el mismo que el de la lista, por lo que debemos tener en cuenta si hay dependencias respecto a otras entidades.
    \item \textit{(Opcional, pero recomendado)} Siguiendo la metodología de pruebas, escribir tests de validación cruzada, que aseguren que los datos importados en ambos sistemas son equivalentes,
    en \verb|Tests/Import/<nombre_script>.py|.
\end{enumerate}

\subsection{Adición de consultas}

Las consultas predefinidas se encuentran en \verb|queries.sql| y \verb|queries.cypher| bajo los directorios \verb|Scripts/PostgreSQL| y \verb|Scripts/Neo4J| respectivamente. Si queremos implementar nuevas consultas, los pasos para ello son:

\begin{enumerate}
    \item Escribir el código SQL y Cypher de la consulta en cada uno de los archivos.
    \item En el fichero \verb|Scripts/database.py|, añadir el nombre de la consulta y sus parámetros al diccionario \code{QUERY\_PARAMETERS}, siguiendo el formato del resto de consultas. Este paso permite lanzar la query desde un script en Python.
    \item \textit{(Opcional, pero recomendado)} Siguiendo la metodología de pruebas, escribir tests de validación cruzada, análisis de casos límite y comprobación de invariantes (propiedades) en el fichero \verb|Tests/<nombre_query>.py|.
\end{enumerate}

\section{Ejecución de pruebas}

Una vez modificado el código, resulta deseable ejecutar pruebas que verifiquen que nuestro cambio no ha tenido efectos inesperados en el resto del sistema. Para ello, podemos utilizar \verb|pytest| (ya instalado como parte de las dependencias del proyecto), y lanzar todos los tests desde el directorio raíz del proyecto:

\begin{verbatim}
$ pytest -v Tests/*.py Tests/Import/*.py
\end{verbatim}

Es importante mencionar que, dado que las pruebas utilizan validación cruzada entre ambos sistemas gestores de bases de datos, debemos asegurarnos de \textbf{haber lanzado ambos, e importado el mismo dataset}.

Para más información acerca de la metodología de pruebas seguida, se puede consultar el capítulo \ref{chapter:tests}.

\section{Errores frecuentes}

\subsection{Las pruebas producen errores de conexión}

Al tratar de ejecutar un script mediante \verb|pytest|, podemos encontrar un mensaje de error similar al siguiente:

\begin{verbatim}
psycopg.OperationalError: connection failed: connection to server
at "127.0.0.1", port 5432 failed: Connection refused
\end{verbatim}

En ese caso, debemos asegurarnos de que los contenedores Docker de PostgreSQL y Neo4J se están ejecutando:

\begin{verbatim}
$ Scripts/PostgreSQL/launch.sh
$ Scripts/Neo4J/launch.sh
\end{verbatim}

También conviene asegurarse de que ningún servicio está tratando de utilizar los mismos puertos, \verb|5432| en el caso de PostgreSQL, y \verb|7474| y \verb|7687| en el de Neo4J.

\subsection{Otros errores}

Dado que durante el uso regular del sistema se llevan a cabo procesos muy similares a los que encontraríamos en el caso del desarrollo, es recomendable consultar el apartado \ref{user:troubleshooting} en el manual de usuario, en que se abordan varios otros errores que podrían aparecer.

\cleardoublepage

\chapter{Manual de usuario} \label{appendix:usage}

\section{Especificaciones} \label{user:specifications}

Se recomienda al usuario contar con un equipo de las siguientes especificaciones, o mejores:

\begin{itemize}
    \item \textbf{CPU:} Intel i7 de octava generación, o su equivalente en AMD.
    \item \textbf{RAM:} 16 GB, dado que el proceso de importación de datos carga mucha información en memoria.
    \item \textbf{Disco duro:} 10 GB libres, en un SSD.
    \item \textbf{Sistema operativo:} Cualquier distribución de Linux, dado que el código proporcionado está diseñado para funcionar con Docker.
\end{itemize}

Cabe destacar que, aunque no se garantiza, es posible que se pueda emplear el sistema en condiciones diferentes, como, por ejemplo, en Windows, o contando con una menor cantidad de RAM.

\section{Dependencias} \label{user:dependencies}

A nivel de software, es necesario contar con los programas indicados en el cuadro \ref{table:dependencias} instalados, y, como recomendación, en las versiones señaladas.

Aunque se hace uso de varios sistemas gestores de bases de datos, no es necesario instalarlos de forma explícita, dado que los scripts proporcionados se encargan de lanzar contenedores Docker que los ejecutan directamente, descargando todo lo requerido.

\begin{table}[ht]
  \centering
  \renewcommand{\arraystretch}{1.2}
  \begin{tabularx}{0.55\textwidth}{Xr}
    \toprule
    \textbf{Nombre} & \textbf{Versión} \\
    \midrule
    Bash & $\texttt{5.2}$ \\
    Coreutils & $\texttt{9.7}$ \\
    Docker & $\texttt{28.1.1}$ \\
    Git & $\texttt{2.49}$ \\
    Python & $\texttt{3.13}$ \\
    QGIS & $\texttt{3.42}$ \\
    \bottomrule
  \end{tabularx}
  \caption{Lista de dependencias software.}
  \label{table:dependencias}
\end{table}


\section{Instalación} \label{user:install}

Para llevar a cabo la instalación del sistema:

\begin{enumerate}
    \item Abrir el terminal y navegar hasta el directorio en que queramos instalar el sistema.
    \item Clonar el repositorio de GitHub haciendo uso del comando:
\begin{verbatim}
$ git clone https://github.com/rodrada/TFG
\end{verbatim}
    \item Asegurarse de que las bibliotecas de Python están instaladas ejecutando, en el directorio raíz del proyecto:
\begin{verbatim}
$ python3 -m venv .venv
$ source .venv/bin/activate
$ pip install -r requirements.txt
\end{verbatim}
    Como resultado, el terminal debería mostrar el prefijo \verb|(.venv)|.
    \item Descargar el dataset que queramos emplear, de webs como \href{https://mobilitydatabase.org}{Mobility Database}. Para esta guía, utilizaremos el de \href{https://mobilitydatabase.org/feeds/gtfs/mdb-1076}{Singapur}.
    \item Descomprimir el dataset y colocarlo en su propio subdirectorio bajo el directorio Datasets/GTFS del proyecto.
    \item Asegurar que el dataset sigue el formato esperado mediante el script process\_dataset.py en el directorio Scripts del proyecto. Por ejemplo:
\begin{verbatim}
$ ./process_dataset.py ../Datasets/GTFS/Singapore
\end{verbatim}
    \item Utilizando los scripts launch.sh en Scripts/Neo4J o Scripts/PostgreSQL, lanzar el sistema gestor de bases de datos que queramos utilizar.
    \item Por último, ejecutar el script import.sh en el directorio Scripts del proyecto, indicando qué dataset queremos importar y en qué sistema gestor de bases de datos. Por ejemplo:
\begin{verbatim}
$ ./import.sh Singapore neo4j
\end{verbatim}
\end{enumerate}

\section{Consultas}

\subsection{Queries predefinidas}

Durante el proceso de importación, se añaden múltiples consultas predefinidas que podemos lanzar de manera sencilla, tal y como se explicará en las próximas subsecciones. En el cuadro \ref{table:consultas_comunes} podemos ver las consultas disponibles en ambos sistemas, y, en el \ref{table:consultas_postgres}, las que solo se pueden ejecutar en PostgreSQL.

% Queries common to both DBMS.
\begin{table}[p]
    \centering
    \begin{tabularx}{\textwidth}{|l|X|}
        \hline
        \textbf{Nombre y argumentos} & \textbf{Descripción} \\
        \hline
        \makecell[lt]{
            \code{active\_services(} \\
            \quad \code{curr\_date: Date} \\
            \code{)} \\
        } &
        Enumera los servicios de transporte activos en la fecha \code{curr\_date}. \\
        \hline
        \makecell[lt]{
            \code{daily\_status(} \\
            \quad \code{start\_date: Date} \\
            \quad \code{end\_date: Date} \\
            \code{)} \\
        } &
        Muestra la cantidad de trayectos, rutas y paradas activos en cada una de las fechas entre \code{start\_date} y \code{end\_date}. \\
        \hline
        \makecell[lt] {
            \code{departure\_times(} \\
            \quad \code{route\_id: String} \\
            \quad \code{stop\_id: String} \\
            \quad \code{curr\_date: Date} \\
            \code{)}
        } &
        Determina los tiempos de salida para la ruta \code{route\_id}, desde la parada \code{stop\_id}, en la fecha \code{curr\_date}. \\
        \hline
        \makecell[lt]{
            \code{headway\_stats(} \\
            \quad \code{curr\_date: Date} \\
            \code{)}
        } &
        Genera estadísticas de los tiempos de espera (percentil 5, mediana, percentil 95 y desviación típica) para todas las rutas en la fecha \code{curr\_date}. \\
        \hline
        \makecell[lt] {
            \code{next\_departures(} \\
            \quad \code{stop\_id: String} \\
            \quad \code{curr\_date: Date} \\
            \quad \code{curr\_time: Time} \\
            \code{)} \\
        } &
        Encuentra las próximas salidas desde la parada \code{stop\_id}, partiendo de la fecha \code{curr\_date} y la hora \code{curr\_time}. \\
        \hline
        \makecell[lt]{
            \code{overlapping\_segments()} \\
        } &
        Busca los segmentos (pares de paradas) de la red que son cubiertos por una mayor cantidad de rutas, lo que puede indicar planificación redundante. \\
        \hline
        \makecell[lt] {
            \code{routes\_by\_relevance(} \\
            \quad \code{curr\_date: Date} \\
            \quad \code{curr\_time: Time} \\
            \code{)} \\
        } &
        Dadas la fecha \code{curr\_date} y la hora \code{curr\_time}, halla las rutas más importantes en términos de trayectos activos y frecuencia entre estos. \\
        \hline
        \makecell[lt]{
            \code{routes\_by\_speed()} \\
        } &
        Produce un ranking de las rutas, ordenadas por su velocidad media (considerando el promedio de todos sus trayectos). \\
        \hline
        \makecell[lt] {
            \code{stops\_within\_distance(} \\
            \quad \code{origin\_lat: Float} \\
            \quad \code{origin\_lon: Float} \\
            \quad \code{seek\_dist: Float} \\
            \code{)}
        } &
        Determina las paradas que se encuentran a menos de \code{seek\_dist} metros del punto con latitud \code{origin\_lat} y longitud \code{origin\_lon}, ordenadas por distancia. \\
        \hline
        \makecell[lt]{
            \code{top\_stops(} \\
            \quad \code{curr\_date: Date} \\
            \code{)} \\
        } &
        Encuentra las paradas más importantes en la fecha \code{curr\_date}, de acuerdo con el número de salidas. \\
        \hline
        \makecell[lt] {
            \code{trip\_start\_time\_distribution(} \\
            \quad \code{curr\_date: Date} \\
            \quad \code{bucket\_size\_min: Integer} \\
            \code{)}
        } &
        Genera un histograma que aglutina las fechas de inicio de los trayectos, en intervalos de \code{bucket\_size\_min} minutos, para la fecha \code{curr\_date}. \\
        \hline    
    \end{tabularx}
    \caption{Consultas disponibles en ambos sistemas.}
    \label{table:consultas_comunes}
\end{table}


% Postgres queries.
\begin{table}[ht]
    \centering
    \begin{tabularx}{\textwidth}{|l|X|}
        \hline
        \textbf{Nombre y argumentos} & \textbf{Descripción} \\
        \hline
        \makecell[lt]{
            \code{stop\_density\_heatmap(} \\
            \quad \code{grid\_size\_meters: Integer} \\
            \code{)} \\
        } &
        Divide el mapa en hexágonos de lado \code{grid\_size\_meters} y determina el número de paradas contenidas en cada uno de ellos. \\
        \hline
        \makecell[lt]{
            \code{earliest\_arrivals(} \\
            \quad \code{origin\_stop\_id: String} \\
            \quad \code{departure\_date: Date} \\
            \quad \code{departure\_time: Time} \\
            \code{)} \\
        } &
        Considerando la parada con identificador \code{origin\_stop\_id}, la fecha \code{departure\_date} y la hora de salida \code{departure\_time}, calcula el instante de llegada más temprano al resto de paradas de la red. \\
        \hline
        \makecell[lt]{
            \code{shortest\_path(} \\
            \quad \code{origin\_stop\_id: String} \\
            \quad \code{destination\_stop\_id: String} \\
            \quad \code{departure\_date: Date} \\
            \quad \code{departure\_time: Time} \\
            \code{)} \\
        } &
        Determina el camino más corto (el que permite llegar antes) desde la parada con identificador \code{origin\_stop\_id} hasta la de identificador \code{destination\_stop\_id}, en la fecha \code{departure\_date} y con la hora de salida \code{departure\_time}. \\
        \hline
        \makecell[lt]{
            \code{route\_straightness()} \\
        } &
        Para cada una de las rutas, determina su índice de rectitud, es decir, el ratio $km\_recorridos / km\_linea\_recta$, lo que puede sugerir ineficiencias (si es reducido) o planificación redundante (si es próximo a 1). \\
        \hline
    \end{tabularx}
    \caption{Consultas exclusivas de PostgreSQL.}
    \label{table:consultas_postgres}
\end{table}

\subsection{PostgreSQL}

Una vez instalado el sistema, podemos empezar a realizar consultas sobre la información almacenada. En el caso de PostgreSQL, procederemos del siguiente modo:

\begin{enumerate}
    \item En el terminal, lanzar el cliente en línea de comandos, psql, utilizando el siguiente comando:
    \begin{verbatim}$ docker exec -it postgres psql -U postgres -d gtfs\end{verbatim}
    Mediante el mismo, se ejecuta una sesión interactiva de psql dentro del contenedor Docker, utilizando el usuario (-U) ``postgres'' y la base de datos (-d) ``gtfs''.
    \item Escribir la consulta que queramos realizar o, en caso de que sea una de las predefinidas,  invocarla utilizando la siguiente línea:
    \begin{verbatim}SELECT * FROM <nombre_consulta>(arg1 => valor1, ...);\end{verbatim}
    También se puede omitir el nombre de los argumentos. Por ejemplo:
    \begin{verbatim}SELECT * FROM stops_within_distance(1.28, 103.82, 500);\end{verbatim}
\end{enumerate}

\begin{figure}[ht]
\centerline{\includegraphics[width=0.8\textwidth]{figuras/postgres_query.png}}
\caption{Ejemplo de consulta en PostgreSQL.}
\label{postgres_query}
\end{figure}

En la figura \ref{postgres_query}, podemos ver un ejemplo de una consulta lanzada en PostgreSQL y el resultado obtenido.

\subsection{Neo4J}

Neo4J ofrece una alternativa más amigable al usuario que la línea de comandos: una interfaz web, visible en la figura \ref{neo4j_webui}, que, por defecto, se lanza en el puerto 7474.

\begin{figure}[ht]
\centerline{\includegraphics[width=0.8\textwidth]{figuras/neo4j_webui.png}}
\caption{Interfaz web de Neo4J.}
\label{neo4j_webui}
\end{figure}

Para llevar a cabo consultas en Neo4J, se debe:

\begin{enumerate}
    \item Lanzar el navegador e introducir la IP de la interfaz web:
    \begin{verbatim}127.0.0.1:7474\end{verbatim}
    \item En la ventana de autenticación que aparece al acceder, pulsar en ``Connect'' \textbf{sin introducir credenciales}.
    \item Abrir la sección ``Database Information'', correspondiente al primer icono en la barra lateral de la izquierda.
    \item Bajo el texto ``Use database'', abrir el desplegable y reemplazar el valor por defecto, \verb|neo4j|, por \verb|gtfs|.
    \item En el cuadro de texto superior, con el mensaje \verb|gtfs$| (véase la figura \ref{neo4j_webui}), introducir el texto de la consulta que queramos realizar. Para lanzar una consulta predefinida, debemos utilizar la siguiente sintaxis:
\begin{verbatim}
MATCH (cq: CypherQuery {name: '<nombre_consulta>'})
CALL apoc.cypher.run(cq.statement, {arg1: valor1, ...})
YIELD value
RETURN value;
\end{verbatim}
    \item Pulsar en el triángulo azul a la derecha del cuadro para lanzar la consulta.
    \item Una vez obtenidos los resultados, seleccionar a la izquierda el modo de presentación de estos: grafo, tabla o texto.
\end{enumerate}


\section{Visualización sobre mapa}

\subsection{QGIS}

Si estamos utilizando PostgreSQL como sistema gestor de bases de datos, podemos visualizar los resultados de nuestras consultas sobre un mapa haciendo uso de QGIS, como indica la figura \ref{qgis_visualization}. Para ello, es necesario:

\begin{enumerate}
    \item Ejecutar QGIS y esperar a que se inicie.
    \item En el menú de la izquierda, hacer click derecho sobre ``PostgreSQL''.
    \item Introducir los siguientes datos de conexión:
    \begin{itemize}
        \item Name: \verb|GTFS|
        \item Host: \verb|127.0.0.1|
        \item Port: \verb|5432|
        \item Database: \verb|gtfs|
        \item SSL mode: \verb|prefer|
        \item Authentication: \verb|Basic|
        \item User name: \verb|postgres| (marcar \verb|store|)
        \item Password: \verb|12345678| (marcar \verb|store|)
    \end{itemize}
    \item Comprobar que los parámetros de conexión se han introducido correctamente pulsando en ``Test Connection'' y, a continuación guardar la conexión pulsando ``OK''.
    \item En el menú de la aplicación, seleccionar $Web > QuickMapServices > Search QMS$.
    \item Buscar, en el menú que se acaba de abrir a la derecha, un proveedor de mapas (se recomienda ``OpenStreetMap''), y hacer doble click en él.
    \item Tras abrirse el mapa de mundo, hacer zoom sobre la ciudad o región asociada al dataset.
    \item Hacer click derecho sobre la conexión que añadimos con anterioridad y pulsar ``Execute SQL...''
    \item Introducir nuestra consulta SQL y lanzarla pulsando en ``Execute''.
    \item Una vez aparezcan los resultados, pulsar en ``Load as a new layer'', abajo a la izquierda. \item Marcar la opción ``Geometry column'' y seleccionar en el desplegable la columna resultado que contenga la geometría que queremos visualizar (contiene un código alfanumérico de 50 caracteres).
    \item Añadir la capa pulsando ``Load Layer'', y asegurarnos de que queda por encima de la capa de OpenStreetMap, en el apartado ``Layers''.
    \item Para encontrar en el mapa los resultados que se han vuelto visibles, hacer click derecho en la capa del resultado, ``QueryLayer'', y pulsar ``Zoom to Layer(s)''.
    \item Si se quiere conservar el mapa y los resultados de consultas, guardar el proyecto, ya sea desde el menú principal de la aplicación, o desde el pop-up que aparece al tratar de cerrarla.
\end{enumerate}

\begin{figure}[p]
    \centerline{\includegraphics[width=0.9\textwidth]{figuras/qgis_visualization.png}}
    \caption{Visualización de múltiples resultados en QGIS.}
    \label{qgis_visualization}
\end{figure}

\subsection{Folium (Python)}

Otra de las opciones para visualizar los resultados es generar gráficas o mapas interactivos, como el de la figura \ref{folium_visualization}, utilizando los scripts de Python proporcionados en el directorio \texttt{Scripts} del proyecto.

Todos ellos proporcionan detalles sobre sus argumentos si los lanzamos con la opción \code{--help}. Por ejemplo, con \code{shortest\_path\_interactive\_map.py}:

\begin{verbatim}
options:
  -h, --help            show this help message and exit
  --origin-stop-id ORIGIN_STOP_ID
                        The ID of the starting stop.
  --destination-stop-id DESTINATION_STOP_ID
                        The ID of the destination stop.
  --date DATE           The departure date in YYYY-MM-DD format.
  --time TIME           The departure time in HH:MI:SS format.
  --output OUTPUT       Path to save the output map HTML file.
\end{verbatim}

\begin{figure}[p]
    \centering
    \includegraphics[width=0.9\textwidth]{figuras/folium_visualization}
    \caption{Visualización de un análisis de alcanzabilidad en Folium.}
    \label{folium_visualization}
\end{figure}

\section{Errores frecuentes} \label{user:troubleshooting}

\subsection{No es posible conectar con el daemon de Docker}

Este problema se caracteriza por el siguiente mensaje al tratar de lanzar alguno de los sistemas gestores de bases de datos:

\begin{verbatim}
docker: Cannot connect to the Docker daemon at
unix:///var/run/docker.sock. Is the docker daemon running?
\end{verbatim}

Se nos indica que Docker ha sido instalado correctamente pero el servicio asociado no se ha lanzado. Para solucionarlo, basta con ejecutar el siguiente comando:
    
\begin{verbatim}
$ sudo systemctl start docker
\end{verbatim}

\subsection{No se puede establecer comunicación con un contenedor Docker}

El mensaje de error que aparece al intentar importar un conjunto de datos es el siguiente:

\begin{verbatim}
Error response from daemon: No such container: <postgres/neo4j>
\end{verbatim}

Al no haberse lanzado el contenedor, el script de importación es incapaz de encontrarlo. Para solucionarlo, desde la carpeta raíz del proyecto, ejecutamos el script launch.sh del sistema gestor de bases de datos correspondiente:

\begin{verbatim}
$ Scripts/<PostgreSQL/Neo4J>/launch.sh
\end{verbatim}

\subsection{Las consultas en Neo4J no producen resultados}

Si al tratar de lanzar una consulta en la interfaz web de Neo4J no se nos muestra ningún nodo como resultado, pero tampoco aparece ningún mensaje de error, es posible que estemos utilizando la base de datos por defecto, \verb|neo4j|, en lugar de la que contiene los datos, llamada \verb|gtfs|.

Para cambiar de base de datos, abrimos la sección ``Database Information'', correspondiente al primer icono en la barra lateral de la izquierda, y, bajo el texto ``Use database'', en el desplegable, elegimos \verb|gtfs|.

\subsection{Los scripts en Python fallan al tratar de importar bibliotecas}

En este caso, al tratar de lanzar un script de importación o generación de mapas interactivos, nos aparece un mensaje de error similar al siguiente (puede mencionar otra biblioteca, no necesariamente la de Neo4J):

\begin{verbatim}
ModuleNotFoundError: No module named 'neo4j'
\end{verbatim}

Como solución, debemos ejecutar los siguientes comandos para lanzar el entorno virtual de Python y obtener las dependencias:

\begin{verbatim}
$ python3 -m venv .venv
$ source .venv/bin/activate
$ pip install -r requirements.txt
\end{verbatim}

\cleardoublepage

\chapter{Licencia} \label{appendix:license}

La parte documental de este trabajo cuenta con la licencia \href{https://creativecommons.org/licenses/by-sa/4.0/deed.es}{Creative Commons Attribution-ShareAlike 4.0}.

A su vez, todo el código fuente se proporciona bajo la licencia \href{https://www.gnu.org/licenses/gpl-3.0.txt}{GPLv3}. Se incluye una copia íntegra de dicha licencia en el repositorio.

\cleardoublepage

\markboth{BIBLIOGRAFÍA}{BIBLIOGRAFÍA}
\addcontentsline{toc}{chapter}{Bibliografía}

\begin{thebibliography}{99}

    % Academic papers.
    \bibitem{dijkstra} E. W. Dijkstra, ``A note on two problems in connexion with graphs'', {\it Numerische Mathematik}, vol. 1, pp. 269–271, 1959.
    \bibitem{csa} Julian Dibbelt, Thomas Pajor, Ben Strasser, and Dorothea Wagner. 2018. Connection Scan Algorithm. ACM J. Exp. Algorithmics 23, Artículo 1.7 (2018), 56 páginas. \url{https://doi.org/10.1145/3274661}

    % Main specs and manuals.
    \bibitem{gtfs} MobilityData, General Transit Feed Specification (GTFS) Schedule Reference. 2025. \url{https://gtfs.org/schedule/reference}
    \bibitem{gtfs-rt} MobilityData, General Transit Feed Specification (GTFS) Realtime Reference. 2025. \url{https://gtfs.org/realtime/reference}
    \bibitem{postgres} The PostgreSQL Global Development Group, PostgreSQL 17 Documentation. 2025. \url{https://www.postgresql.org/docs/17}
    \bibitem{postgis} PostGIS Project Steering Committee, PostGIS 3.5 Manual. 2025. \url{https://postgis.net/docs/manual-3.5}
    \bibitem{neo4j} Neo4j, Inc., Neo4j Graph Database Operations Manual, v5, 2025. \url{https://neo4j.com/docs/operations-manual/5}
    \bibitem{cypher} Neo4j, Inc., The Cypher Query Language Reference, v5, 2025. \url{https://neo4j.com/docs/cypher-manual/5}
    \bibitem{apoc} Neo4j, Inc., APOC Extended Documentation, v5, 2025. \url{https://neo4j.com/labs/apoc/5}
    \bibitem{neo4j-spatial} Neo4j, Inc., Neo4j-Spatial User Guide, v5, 2025. \url{https://neo4j.com/labs/neo4j-spatial/5}

    % Additional software documentation.
    \bibitem{docker} Docker Inc., Docker Engine User Guide. 2025. \url{https://docs.docker.com/engine}
    \bibitem{python} Python Software Foundation, Python 3.13 Documentation. 2025. \url{https://docs.python.org/3.13}
    \bibitem{pytest} H. Krekel et al., pytest: helps you write better programs, v8.4.x, 2025. \url{https://docs.pytest.org/en/8.4.x}
    \bibitem{hypothesis} D. R. MacIver et al., Hypothesis: Property-based testing for Python. 2025. \url{https://hypothesis.readthedocs.io}
    \bibitem{folium} Folium Developers, Folium: Python data, leaflet.js maps, v0.20.0, 2025. \url{https://python-visualization.github.io/folium}
    \bibitem{pandas} The pandas development team, pandas documentation, v2.3, 2025. \url{https://pandas.pydata.org/pandas-docs/version/2.3}
    \bibitem{qgis} QGIS Development Team, QGIS User Guide, v3.40, 2025. \url{https://www.qgis.org/resources/hub}

    % Wikipedia articles.
    \bibitem{betweenness} Wikipedia, Betweenness centrality. Artículo de wikipedia (\url{https://en.wikipedia.org/wiki/Betweenness_centrality}). Consultado en noviembre de 2025.

\end{thebibliography}



\end{document}
