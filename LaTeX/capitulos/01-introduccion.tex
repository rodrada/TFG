
\chapter{Introducción}

La gestión y el análisis de datos almacenados en redes geoespaciales se han convertido en parte crucial de múltiples dominios, destacando especialmente en el sector del transporte público. A causa de la creciente complejidad de dichas redes, que combinan información geográfica y temporal, se vuelve necesario el desarrollo de soluciones robustas y eficientes, que a su vez sean capaces de representar de manera conveniente las relaciones entre elementos como paradas, rutas y horarios.

En este contexto, las tecnologías de bases de datos desempeñan un papel crucial. Aunque los sistemas gestores de bases de datos relacionales, es decir, SQL, han sido tradicionalmente la opción dominante por un amplio margen, el surgimiento de los nuevos paradigmas NoSQL, en particular las bases de datos orientadas a grafos, ofrece enfoques alternativos. Se parte de la hipótesis de que estas últimas podrían ofrecer ventajas significativas en eficiencia y expresividad de consultas, al estar centradas en el modelado de un gran número de relaciones y la recuperación de información mediante la navegación a través de nodos.

Este trabajo aborda precisamente el reto de gestionar redes geoespaciales de transporte público, explorando y comparando dos aproximaciones tecnológicas diferentes: un sistema relacional consolidado como PostgreSQL, junto a su extensión geoespacial PostGIS, frente a una de las opciones mejor establecidas en el ámbito de las bases de datos de grafos, Neo4J, utilizando la extensión Neo4J Spatial. En ambos casos se ha optado por tecnologías estables y maduras para garantizar la fiabilidad de los sistemas.

En cuanto a los conjuntos de datos empleados, estos siguen el estándar de facto \textbf{GTFS} (General Transit Feed Specification), y representan redes a diferentes escalas, incluyendo tanto entornos urbanos (Bogotá, Praga, Hong Kong...), como regionales (por ejemplo, Galicia), a fin de poder considerar la escalabilidad que cada sistema ofrece.

Dado que no existe una manera simple de trasladar la información, se busca desarrollar mecanismos para \textbf{importar y modelar} las redes en ambos sistemas gestores de bases de datos y, posteriormente, implementar un conjunto representativo de \textbf{consultas} pensadas para satisfacer necesidades de información típicas, ya sean del usuario final o de las entidades gestoras del transporte público.

Una vez recuperada la información necesaria, el siguiente paso es presentar los resultados de manera amigable a los usuarios, y, para lograr esto, proporcionar una solución para la \textbf{visualización} de los mismos, a través del software QGIS y de la biblioteca Folium, de Python, facilitando su interpretación sobre representaciones cartográficas.

En último lugar, se realiza una \textbf{evaluación comparativa} de los dos sistemas, que no se limita a métricas cuantitativas, es decir, el rendimiento, sino que también incluye apartados cualitativos como la facilidad para modelar la red geoespacial y la comodidad a la hora de expresar las consultas en el lenguaje de cada sistema. Con ello, se busca determinar las fortalezas e inconvenientes de cada una de las alternativas y clarificar en qué ocasiones resulta preferible su uso.

A lo largo del resto de la memoria se profundiza en diversos aspectos del trabajo. Tras esta introducción, se definen las necesidades de información y funcionalidades relevantes para los usuarios y gestores del sistema de transporte público (capítulo \ref{chapter:requirements}), se detalla la solución propuesta, tanto a nivel de modelado como de importación y consulta (capítulo \ref{chapter:design}), se documenta el plan de pruebas seguido para validar los requisitos previamente planteados (capítulo \ref{chapter:tests}), se realiza una evaluación comparativa tanto del rendimiento como de la facilidad de expresión de consultas en cada uno de los sistemas (capítulo \ref{chapter:comparison}) y, finalmente, se presentan las conclusiones obtenidas a raíz de la comparación entre ambos sistemas, así como líneas de trabajo futuras para ampliar el alcance del proyecto (capítulo \ref{chapter:conclusion}).

Por otra parte, la memoria se complementa con tres anexos: el manual técnico, para las personas que modifiquen el sistema o lleven a cabo algún desarrollo posterior (anexo \ref{appendix:development}), el manual de usuario, que explica cómo instalar, configurar y utilizar el software (anexo \ref{appendix:usage}), y la licencia (anexo \ref{appendix:license}).
