\pagestyle{plain}
\chapter*{Resumen}
En este trabajo se han modelado redes geoespaciales de transporte público en dos sistemas gestores de bases de datos diferentes, PostgreSQL y Neo4J, partiendo del estándar de modelado GTFS. Después, se ha desarrollado un conjunto representativo de consultas que responden a necesidades de información reales, tanto de usuarios de la red como de las empresas que ofrecen dichos servicios.

A continuación, se ha trabajado en la representación gráfica, sobre el globo terráqueo, de los resultados generados, utilizando QGIS y Folium, y, en último lugar, se ha llevado a cabo una comparativa entre las prestaciones de ambos sistemas, centrada no solo en la eficiencia computacional, sino también en la facilidad de modelado y expresión de consultas.
 